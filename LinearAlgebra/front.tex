%!TEX root = larxxia.tex
%\frontmatter

\maketitle

\begin{comment}
\addtocounter{page}{-1}
In this draft, the blue text comments such as this are notes to developers about related references, about reasons for decisions, about more exercises, about possible future extensions, and so on.
Such comments are not intended for the published version, they are only notes for myself and developers.
\begin{quoted}{George Cobb, 2007}
the consensus curriculum is still an unwitting prisoner of history
\end{quoted}


Outstanding tasks for a first version include:
\begin{itemize}
\item finalising the scope of the book and of applications;
\item exercises on computing that do not require a computer, e.g., interpretation;
\item checking the staged use of abstract symbolism; 
\item potentially more applications that involve `real' data, especially in Chapter~5 on approximating matrices;
\item adapt information, especially some uses of the SVD, from the book by Mark Holmes (2016) ``Introduction to scientific computing and data analysis'' Springer;
\item refining the index;
\item possibly concept maps;
\item short videos of procedures, examples, proofs.
\end{itemize}
\end{comment}



\tableofcontents


\chapter*{Preface}


\begin{quoted}{\index{Moler, Cleve}Cleve Moler, MathWorks (2006)}
Traditional courses in linear algebra make considerable use of the 
{reduced row echelon form} (\rref), but the \rref\ is an unreliable tool for computation in the face of inexact data and arithmetic. 
The [Singular Value Decomposition] \svd\ can be regarded as a modern, computationally powerful replacement for the \rref.\footnote{\url{http://au.mathworks.com/company/newsletters/articles/professor-svd.html} [9 Jan 2015]}
\end{quoted}

The Singular Value Decomposition (\svd) is sometimes called the \emph{\idx{jewel in the crown}} of linear algebra.
Traditionally the \svd\ is introduced and explored at the end of several linear algebra courses.
Question: Why were students required to wait until the end of the course, if at all, to be introduced to beauty and power of this jewel?
Answer: limitations of hand calculation.

This book establishes a new route through linear algebra, one that reaches the \svd\ jewel in linear algebra's crown very early, in \autoref{sec:fisvd}.
Thereafter its beautiful power both explores many modern applications and also develops traditional linear algebra concepts, theory, and methods.
No rigour is lost in this new route: indeed, this book demonstrates that most theory is better proved with an \svd\ rather than with the traditional \rref.
This new route through linear algebra becomes available by the ready availability of ubiquitous computing in the 21st century.


\begin{quoted}{\cite[p.30]{Turner2014}}
As so many other disciplines use the \svd, it is not only important that mathematicians understand what it is, but also teach it thoroughly in linear algebra and matrix analysis courses.
\end{quoted}











\section*{Aims for students}

\begin{quoted}{\cite{Bressoud2014}}
How should mathematical sciences departments reshape their curricula to suit the needs of a well-educated workforce in the twenty-first century?

\ldots\
The mathematical sciences themselves are changing as the needs of big data and the challenges of modeling complex systems reveal the limits of traditional curricula.
%
%\ldots\
%more closely intertwine the learning of mathematics with the appreciation of its applications.
\end{quoted}



Linear algebra is packed with compelling results for application in science, engineering and computing, and with answers for the twenty-first century needs of big data and complex systems.
This book provides the conceptual understanding of the essential linear algebra of vectors and matrices for modern engineering and science.
The traditional linear algebra course has been reshaped herein to meet modern demands.

Crucial is to inculcate the terms and corresponding relationships that you will most often encounter later in professional life, often when using professional software.  
For example, the manual for the engineering software package Fluent most often invokes the linear algebra terms of diagonal, dot product, eigenvalue, least square, orthogonal, projection, principal axes, symmetric, unit vector.
Engineers need to know these terms.
What such useful terms mean, their relationships, and use in applications are central to the mathematical development in this book: you will see them introduced early and used often.

For those who proceed on to do higher mathematics, the development also provides a great solid foundation of key concepts, relationships and transformations necessary for higher mathematics. 

Important for all is to develop facility in manipulating, interpreting and transforming between visualisations, algebraic forms, and vector-matrix representations---of problems, working, solutions and interpretation.
In particular, one overarching aim of the book is to encourage your formulation, thinking and operation at the crucial system-wide level of matrix\slash vector operations.

In view of ubiquitous computing, throughout this book explicitly integrates computer support for developing concepts and their relations.
The central computational tools to understand are the operation~\verb|A\| for solving straightforward linear equations; the function~\verb|svd()| for difficult linear equations; the function~\verb|svd()| also for approximation; and function~\verb|eig()| for probing structures.
This provides a framework to understand key computational tools to effectively utilise the so-called third arm of science: namely, computation. 

Throughout the book examples (many graphical) introduce and illustrate the concepts and relationships between the concepts.
Working through these will help form the mathematical relationships essential for application.
Interspersed throughout the text are questions labelled ``Activity'': these aim to help form and test your understanding of the concepts being developed.

Also included are many varied applications, described to varying levels of details.
These applications indicate how the mathematics will empower you to answer many practical challenges in engineering and science.


\begin{quoted}{\cite[p.xiii]{Arnold2014}}
The main contribution of mathematics to the natural sciences is not in formal computations \ldots, but in the investigation of those non-formal questions where the exact setting of the question (what are we searching for and what specific models must be used) usually constitute half the matter.
%
%\ldots\
%Examples teach no less than rules, and errors, more than correct but abstruse proofs.  
%Looking at the pictures in this book, the reader will understand more than learning by rote dozens of axioms
\end{quoted}








\section*{Background for teachers}

Depending upon the background of your students, your class should pick up the story somewhere in the first two or three chapters.
Some students will have previously learnt some vector material in just 2D and 3D,  in which case refresh the concepts in~\(n\)D.
% as presented in the first chapter.

As a teacher you can use this book in several ways.
\begin{itemize}
\item One way is as a reasonably rigorous mathematical development of concepts and interconnections by invoking its definitions, theorems, and proofs, all interleaved with examples.
\item Another way is as the development of practical techniques and insight for application orientated science and engineering students via the motivating examples to appropriate definitions, theorems and applications. 
\item Or any mix of these two.
\end{itemize}

One of the aims of this book is to organise the development of linear algebra so that if a student only studies part of the material, then s/he still obtains a powerful and useful body of knowledge for science or engineering.

The book typically introduces concepts in low dimensional cases, and subsequently develops the general theory of the concept.  
This is to help focus the learning, empowered by visualisation, while also making a preformal connection to be strengthened subsequently.
People are not one dimensional; knowledge is not linear.
Cross-references make many connections, explicitly recalling earlier learning (although sometimes forward to material not yet `covered').
\begin{quoted}{\cite{Halpern2003} [p.38]}
information that is recalled grows stronger with each retrieval \ldots\ spaced practice is preferable to massed practice.
\end{quoted}

One characteristic of the development is that the concept of linear independence does not appear until relatively late, namely in \autoref{ch:gee}.
This is good for several reasons.
First, orthogonality is much more commonly invoked in science and engineering than is linear independence.
Second, it is well documented that students struggle with linear independence:
\begin{quoted}{\cite{Uhlig02}}
there is ample talk in the math ed literature of classes hitting a `brick wall', when linear (in)dependence is studied in the middle of such a course
\end{quoted}
Consequently, here we learn the more specific orthogonality before the more abstract linear independence.
Many modern applications are made available by the relatively early introduction of orthogonality.

In addition to many exercises at the end of every section, throughout the book are questions labelled ``Activity''.
These are for the students to do to help form the concepts being introduced with a small amount of work (currently, each version of this book invokes a different (random) permutation of the answers).
These activities may be used in class to foster active participation by students (perhaps utilising clickers or web tools such as that provided by \url{http://www.quizsocket.com}).
Such active learning has positive effects \cite[]{ED498555}.




\subsection*{On visualisation}

\begin{quoted}{\cite[p.38]{CUPMguide2015}}
All Linear Algebra courses should stress visualization and geometric interpretation of theoretical ideas in 2- and 3-dimensional spaces. Doing so highlights ``algebraic and geometric'' as ``contrasting but complementary points of view,''
\end{quoted}


Throughout, this book also integrates visualisation.
This visualisation reflects the fundamentally geometric nature of linear algebra.  
It also empowers learners to utilise different parts of their brain and integrate the knowledge together from the different perspectives.
Visualisation also facilitates greater skills at interpretation and modelling so essential in applications.
But as commented by \cite{Fara2009} [p.249] ``just like reading, deciphering graphs and maps only becomes automatic with practice.''
Lastly, visual exercise questions develops understanding without a learner being able to defer the challenge to online tools, as yet.  

\begin{quoted}{\cite{Moody2009}}
Visual representations are effective because they tap into the capabilities of the powerful and highly parallel human visual system.
We like receiving information in visual form and can process it very efficiently: around a quarter of our brains are devoted to vision, more than all our other senses combined  \ldots
researchers (especially those from mathematic backgrounds) see visual notations as being informal, and that serious analysis can only take place at the level of their semantics. 
However, this is a misconception: visual languages are no less formal than textual ones
\end{quoted}









\subsection*{On integrated computation}

\begin{quoted}{\cite[p.38]{CUPMguide2015}}
Cowen argued that because ``no serious application of linear algebra happens without a computer,'' computation should be part of every beginning Linear Algebra course. \ldots\
While the increasing applicability of linear algebra does not require that we stop teaching theory, Cowen argues that ``it should encourage us to see the role of the theory in the subject as it is applied.''
\end{quoted}


We need to empower students to use computers to improve their understanding, learning and application of mathematics; not only integrated in their study but also in their later professional career.

One often expects it should be easy to sprinkle a few computational tips and tools throughout a mathematics course.
This is not so---extra computing is difficult.
There are two reasons for the difficulty: 
first, the number of computer language details that have to be learned is surprisingly large;
second, for students it is a genuine intellectual overhead to learn and relate both the mathematics and the computations.

Consequently, this book chooses a computing language where it is as simple as reasonably possible to perform linear algebra operations: \script\ appears to answer this criteria.\footnote{To compare popular packages, just look at the length of expressions students have to type in order to achieve core computations: \script\ is almost always the shortest \cite[e.g.]{Nakos1998}.  
(Of course be wary of this metric: e.g., \textsc{apl} would surely be too concise!)}
Further, we are as ruthless as possible in invoking herein the smallest feasible set of commands and functions from \script\ so that  students have the minimum to learn.
Most teachers will find many of their favourite commands are missing---this omission is all to the good in focussing upon useful mathematical development aided by only essential integrated computation.

This book does not aim to teach computer programming: there is no flow control, no looping, no recursion, nor function definitions.
The aim herein is to use short sequences of declarative assignment statements, coupled with the power of vector and matrix data structures, to learn core mathematical concepts, applications and their relationships in linear algebra. 

The internet is now ubiquitous and pervasive. 
So too is computing power: students can execute \script\ not only on laptops, but also on tablets and smart phones, perhaps using university or public servers, \verb|octave-online.net|, \script[1]-Online or \script[1]-Mobile. 
We no longer need separate computer laboratories.
Instead, expect students to access computational support simply by reaching into their pockets or bags.

\begin{quoted}{\cite{Donoho2015}}
long after Riemann had passed away, historians discovered that he had developed advanced techniques for calculating the Riemann zeta function and that his formulation of the Riemann hypothesis---often depicted as a triumph of pure thought---was actually based on painstaking numerical work.
\end{quoted}







\section*{Linear algebra for statisticians}

This book forms an ideal companion to modern statistics courses.
The recently published Curriculum Guidelines for Undergraduate Programs in Statistical Science by \cite{StatsEduGuidelines2014} emphasises that linear algebra courses must provide ``matrix manipulations, linear transformations, projections in Euclidean space, eigenvalues\slash eigenvectors, and matrix decompositions''.
These are all core topics in this book, especially the statistically important \svd\ factorisation (Chapts.~\ref{ch:m} and~\ref{ch:am}).
Furthermore, this book explicitly makes the recommended ``connections between concepts in these mathematical foundations courses and their applications in statistics'' \cite[p.12]{StatsEduGuidelines2014}.

Moreover, with the aid of some indicative statistical applications along with the sustained invocation of ``visualization'' and  ``basic programming concepts'', this book helps to underpin the requirement to ``Encourage synthesis of theory, methods, computation, and applications'' \cite[p.13]{StatsEduGuidelines2014}.






\section*{Acknowledgements}

I acknowledge with thanks the work of many others who inspired much design and details here, including 
the stimulating innovations of calculus reform \cite[e.g.]{HughesHallett2013},  
the comprehensive efforts behind recent reviews of  undergraduate mathematics and statistics teaching \cite[e.g.]{Alpers2013, Bressoud2014, Turner2014, StatsEduGuidelines2014, CUPMguide2015, gaimme2016}, 
and the books of \cite{Anton6, Davis99a, Holt2013, Larson2013, Lay2012, Nakos1998, Poole2015, Will04}.
I also thank the entire \LaTeX\ team, especially Knuth, Lamport, Feuers\"anger, and the \textsc{ams}.








