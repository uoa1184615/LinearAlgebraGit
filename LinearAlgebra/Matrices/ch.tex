%!TEX root = ../larxxia.tex

\chapter{Matrices encode system interactions}
\label{ch:m}

\minitoc




\autoref{sec:dmsls} introduced matrices in the matrix-vector form \(A\xv=\bv\) of a system of linear equations.
This chapter starts with \cref{sec:moaa,sec:im} developing the basic operations on matrices that make them so useful in applications and theory---including making sense of the `product'~\(A\xv\).
\autoref{sec:fisvd} then explores how the so-called ``singular value decomposition (\svd)'' of a matrix empowers us to understand how to solve general linear systems of equations, and a graphical meaning of a matrix in terms of rotations and stretching.
The structures discovered by an \svd\ lead to further conceptual development (\autoref{sec:sbd}) that underlies the at first paradoxical solution of inconsistent equations (\autoref{sec:asie}).
Finally, \autoref{sec:ilt} unifies the geometric views invoked.



\begin{quoted}{Wigner, 1960 \cite[p.3]{Mandelbrot1982}}
the language of mathematics reveals itself unreasonably effective in the natural sciences \ldots\ a wonderful gift which we neither understand nor deserve.  We should be grateful for it and hope that it will remain valid in future research and that it will extend, for better or for worse, to our pleasure even though perhaps also to our bafflement, to wide branches of learning
\end{quoted}


