%!TEX root = ../larxxia.tex
\begin{draft}
\section{Define advanced vector spaces}
\label{sec:davsof}

\secttoc

\begin{comment}
axiomatic approach.  Rebuild the foundations.
Maybe \pooliv{Ch.6 and~7} \larsvii{Ch.5 and~8} \holti{Ch.10, \S11.4--5}
\end{comment}

To take full advantage of the power of linear algebra, we extend the definition of what can be a ``vector''.
\begin{aside}
This advanced abstract algebra may be learnt anytime after \autoref{sec:dpdal}.
\end{aside}
In particular we extend analysis to complex-valued vectors, and to functions as vectors.
To do such extension, instead of `patching in' case-by-case variations, we return to the fundamentals and define what we mean by a vector through its properties.
By its properties we will know a ``vector'', not by any preconceptions about shape, form or appearance.


%
%\subsection{Fields generalise numbers}
%\label{sec:fgn}
%
%The components of vectors are commonly real or complex numbers.
%But in the spirit of removing preconceptions, let's not assume so.
%Instead let's allow the scope of components to be even more general.
%What we need is a set of `values' on which are defined operations that behave like addition, subtraction, multiplication and division for real numbers.
%Such a set together with its operations are called a \emph{\idx{field}}.
%
%However, except where otherwise specified, in these advanced sections the default field is the field of complex numbers,~\(\CC\).
%
%The following fundamental definition introduces a field as a set with addition and multiplication.
%Then subtraction and division arise as the inverse of addition and multiplication, respectively.
%
%\begin{definition}[field] \label{def:field}
%A \bfidx{field} is a set of objects, say denoted~\(\FF\), together with two binary operations called \bfidx{addition} and \bfidx{multiplication}, often denoted by symbols~\(+\) and~\(\cdot\), respectively.
%Further, the operations must satisfy the following properties for every  \(a,b,c\in \FF\):
%\begin{enumerate}
%\item \bfidx{associativity} of both addition and multiplication: \(a + (b + c) = (a + b) + c\) and \(a \cdot (b \cdot c) = (a \cdot b) \cdot c\)\,;
%\item \bfidx{commutativity} of both addition and multiplication: \(a + b = b + a\) and \(a \cdot b = b \cdot a\)\,;
%\item \bfidx{identity} for both addition and multiplication: there exist two distinct elements in the field~\(\FF\), denoted~\(0\) and~\(1\), such that \(a + 0 = a\) and \(a \cdot 1 = a\)\,;
%\item \bfidx{inverse}, for every \(a\in \FF\): 
%\begin{itemize}
%\item there exists an element in the field~\(\FF\), denoted~\(-a\), called an additive inverse of~\(a\), such that \(a + (-a) = 0\);
%\item when \(a\neq 0\) there exists an element in the field~\(\FF\), denoted by \(a^{-1}\) or~\(1/a\), called a multiplicative inverse of~\(a\), such that \(a \cdot a^{-1} = 1\);
%\end{itemize}
%\item \bfidx{distributivity} of multiplication over addition: \(a \cdot (b + c) = (a \cdot b) + (a \cdot c)\).
%\end{enumerate}
%\end{definition}
%
%
%\begin{comment}
%reals, complex, boolean, \(GF4\) are fields.  Integers are not, but clock arithmetic is.  Matrices are not.  3-vectors are not.
%\end{comment}
%
%
%\subsubsection{The field of complex numbers}
%\begin{comment}
%\CC\ is a field. With \((a,b)\) usually denoted by \(a+\i b\) (\(\i b=0+\i b\) and \(a=a+\i 0\)).
%Complex conjugate, absolute value, inverse, De Moivre.
%\end{comment}
%
%
%
%\subsubsection{General fields}
%\begin{comment}
%Most properties are very familiar for real or complex numbers, so may skip the rest of this subsection if that is all you need.
%Inverses and identities are unique;
%\((-1)\cdot x=-x\);
%if \(a+b=c+b\) then \(a=c\);
%if \(ab=cb\) and \(b\neq 0\) then \(a=c\);
%\(a\cdot 0=0\);
%\((-a)b=a(-b)=-(ab)\);
%\((-a)(-b)=ab\)
%\end{comment}
%

\subsection{The field of complex numbers}
\label{sec:fcn}

Recall from the start of \autoref{ch:v} that the term \idx{scalar} refers to a number that could be integer, real or complex, but usually was a real number.
But advanced algebra often need complex numbers as the main numbers in applications, especially engineering and physics. 
We need to cater for applications where complex numbers are the default.
Consequently, in these sections on advanced algebra the term scalar will usually mean a complex number.

Then real numbers and integers are useful special cases encompassed by the complex numbers.

Because complex numbers are the default for these advanced sections, this subsection overviews the fundamental properties of complex numbers.
\begin{aside}
Bypass this subsection if familiar with complex numbers and their basic operations.
\end{aside}

\begin{definition}\label{def:cc}
In terms of real numbers, a \bfidx{complex number} is an ordered pair of real numbers, say~\(a\) and~\(b\), written as \(a+b\i\), or~\(b\i\) when \(a=0\), or~\(a\) when \(b=0\).
The set of all complex numbers is denoted~\index{C@\CC|textbf}\CC, and sometimes called the \idx{field}~\CC.
\end{definition}

Arithmetic operations on complex numbers have much the same properties as those for real numbers.
We start with the fundamental properties for addition and multiplication, then these lead to subtraction and division of complex numbers.
But complex numbers have further properties summarised at the end.


\begin{theorem} \label{def:ccfield}
For every two complex numbers \(x=a+b\i\) and \(y=c+d\i\) define 
the two binary operations of \bfidx{addition} and \bfidx{multiplication} 
as, respectively,
\begin{itemize}
\item \(x+y=(a+b\i)+(c+d\i):=(a+c)+(b+d)\i\),
\item \(xy=x\cdot y=(a+b\i)\cdot(c+d\i):=(ac-bd)+(ad+bc)\i\).
\end{itemize}
These operations satisfy the following properties for every  \(x,y,z\in \CC\):
\begin{enumerate}
\item \bfidx{associativity} of both addition and multiplication: \(a + (b + c) = (a + b) + c\) and \(a \cdot (b \cdot c) = (a \cdot b) \cdot c\)\,;
\item \bfidx{commutativity} of both addition and multiplication: \(a + b = b + a\) and \(a \cdot b = b \cdot a\)\,;
\item \bfidx{identity} for both addition and multiplication: there exist two distinct elements in the field~\(\CC\), denoted~\(0\) and~\(1\), such that \(a + 0 = a\) and \(a \cdot 1 = a\)\,;
\item \bfidx{inverse}, for every \(a\in \CC\): 
\begin{itemize}
\item there exists an element in the field~\(\CC\), denoted~\(-a\), called an additive inverse of~\(a\), such that \(a + (-a) = 0\);
\item when \(a\neq 0\) there exists an element in the field~\(\CC\), denoted by \(a^{-1}\) or~\(1/a\), called a multiplicative inverse of~\(a\), such that \(a \cdot a^{-1} = 1\);
\end{itemize}
\item \bfidx{distributivity} of multiplication over addition: \(a \cdot (b + c) = (a \cdot b) + (a \cdot c)\).
\end{enumerate}
\end{theorem}




\subsection{Vector spaces over a field}

Previously, a vectors was defined to be an \(n\)-tuple of numbers.
But now we focus on the essential properties of such vectors in order to free ourselves for more general development and applications.

%The previous \autoref{sec:fgn} extended the idea of real and complex numbers to general fields.  
%For ease of discussion we define ``scalar'' to encompass generalised numbers.

%\begin{definition}[scalar] \label{def:vscalar}
%For any given \idx{field}~\(\FF\), the term \bfidx{scalar} refers to any element in~\(\FF\).
%\end{definition}



\begin{definition}[vector space] \label{def:vsf}
A \bfidx{vector space} over the field~\(\CC\) is a set~\(\VV\) \emph{together} with two binary operations called \bfidx{vector addition} and \bfidx{scalar multiplication}.
Elements in~\VV\ are called \bfidx{vector}s, and are often denoted by a boldface font.
Further, the operations must satisfy the following for every vector \(\uv,\vv,\wv\in\VV\) and every scalar~\(a\) and~\(b\):
\begin{enumerate}
\item \idx{vector addition} of vectors~\uv\ and~\vv\ results in a vector, denoted \(\uv+\vv\);
\item \idx{scalar multiplication} of vector~\uv\ by scalar~\(a\) results in a vector, denoted~\(a\uv\);
\item associativity of addition 	\(\uv + (\vv + \wv) = (\uv + \vv) + \wv\);
\item \bfidx{commutativity} of addition, \(\uv + \vv = \vv + \uv\);
\item \bfidx{identity} element of addition, there exists an element \(\ov\in\VV\), called the \bfidx{zero vector}, such that \(\vv + \ov = \vv\);
\item \bfidx{inverse} of addition, there exists an element \(-\vv\in\VV\)  such that \(\vv + (-\vv) = \ov\);
\item \bfidx{compatibility} of multiplication, 	\(a(b\vv) = (ab)\vv\);
\item \bfidx{identity} element of scalar multiplication, \(1\vv = \vv\), for the multiplicative identity~\(1\) in the field~\(\CC\).
\item \bfidx{distributivity} of scalar multiplication over vector addition, \(a(\uv + \vv) = a\uv + a\vv\);
\item \bfidx{distributivity} of scalar multiplication over scalar addition, \((a + b)\vv = a\vv + b\vv\).
\end{enumerate}
\end{definition}




\begin{theorem}\label{thm:avssvx}
For every vector~\uv\ in a vector space~\VV, and for every~\(c\) in the underlying field scalar, the following hold:
\begin{enumerate}
\item \(0\uv=\ov\);
\item \(c\ov=\ov\);
\item \((-1)\uv=-\uv\); 
\item if \(c\uv=\ov\), then either \(c=0\) or \(\uv=\ov\).
\end{enumerate}
\end{theorem}



\begin{definition}[subspace] \label{def:avsss} % x-ref to other subspace??
Given a vector space~\VV, a subset~\WW\ of~\VV\ is called a \bfidx{subspace} of~\VV\ if~\WW\ is itself a \idx{vector space} with the same \idx{vector addition} and \idx{scalar multiplication} as that for~\VV.
\end{definition}


\begin{theorem} \label{thm:avsss} % x-ref?? or omit and prove instead
Given a vector space~\VV, and a non-empty subset~\WW\ of~\VV,
the subset~\WW\ is a \idx{subspace} if and only if both the following two closure conditions hold:
\begin{enumerate}
\item for every \(\uv,\vv\) in~\WW, \(\uv+\vv\) is in~\WW;
\item for every \uv~in\WW\ and every scalar~\(c\), \(c\uv\) is in~\WW.
\end{enumerate}
\end{theorem}


\begin{comment}
\ov\ is in every subspace.
Spanning sets.
\end{comment}




\sectionExercises

\end{draft}
