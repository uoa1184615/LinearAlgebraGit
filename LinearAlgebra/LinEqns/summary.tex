%!TEX root = ../larxxia.tex

\section{Summary of linear equations}
\label{sec:sumle}

\begin{itemize}
\def\index#1{}% turn off indexing

\subsubsection{Introduction to systems of linear equations}


\itemhi A \bfidx{linear equation} in the \(n\)~variables \(x_1,x_2,\ldots,x_n\) is an equation that can be written in the form (\cref{def:lineqn})
\begin{equation*}
\lincomb axn=b\,.
\end{equation*}
A \bfidx{system} of linear equations is a set of one or more linear equations in one or more variables.


\item Algebraic manipulation, such as that for the \gps, can sometimes extract a tractable system of linear equations from an intractable nonlinear problem.
Often the algebraic manipulation forms equations in an abstract setting where it is difficult to interpret the mathematical quantities---but the effort \text{is worthwhile.}



\subsubsection{Directly solve linear systems}

\itemme The \bfidx{matrix-vector form} of a given system is \(A\xv=\bv\) (\cref{def:matvecsys}) for the \(m\times n\) \bfidx{matrix} of coefficients
\begin{equation*}
A=\begin{bmatrix} a_{11}&a_{12}&\cdots&a_{1n}
\\a_{21}&a_{22}&\cdots&a_{2n}
\\\vdots&\vdots&\ddots&\vdots
\\a_{m1}&a_{m2}&\cdots&a_{mn} \end{bmatrix},
\end{equation*}
and vectors \(\xv=(\hlist xn)\) and \(\bv=(\hlist bm)\).
If \(m=n\), then \(A\)~is called a \bfidx{square matrix}.

\itemhi \cref{pro:unisol} use \script\ to solve the matrix-vector system \(A\xv=\bv\), for a \idx{square matrix}~\(A\): 
\begin{enumerate}
\item form matrix~\(A\) and \idx{column vector}~\(\bv\);
\item check \verb|rcond(A)| exists and is not too small, \(1\geq\text{good} >10^{-2} >\text{poor} >10^{-4} >\text{bad} >10^{-8} >\text{terrible}\);
\item if \verb|rcond(A)| is acceptable, then execute \verb|x=A\b| to compute the solution vector~\(\xv\). 
\end{enumerate}
Checking \verb|rcond()| avoids many mistakes in applications.


\item In \script\ you may need the following.
\begin{itemize}
\itemhi \verb|[ ... ; ... ; ... ]| forms both matrices and vectors, or use newlines instead of the semi-colons.
\itemhi \verb|rcond(A)|  of a \idx{square matrix}~\(A\) \emph{estimates} the reciprocal of the so-called \idx{condition number}.
\itemhi \verb|x=A\b| computes an `answer' to \(A\xv=\bv\)\,---but it may not be a solution unless \verb|rcond(A)| exists and is not small;
\item Change an element of an array or vector by assigning a new value with assignments \verb|A(i,j)=...| or \verb|b(i)=...| where \verb|i| and~\verb|j| denote some indices.
\item For a vector (or matrix) \verb|t| and an exponent~\verb|p|, the operation \verb|t.^p| computes the \verb|p|th~power of each element in the vector.
\itemme The function \index{ones()@\texttt{ones()}}\verb|ones(m,1)| gives a (column) vector of \(m\)~ones, \((1,1,\ldots,1)\).
\end{itemize}

\item For fitting data, generally avoid using polynomials of degree higher than cubic.


\item To solve systems by hand we need several more notions.
The \bfidx{augmented matrix} of the \idx{system} \(A\xv=\bv\) is the matrix \(\begin{bmatrix} A\V \bv \end{bmatrix}\)  (\cref{def:augmat}).

\item \bfidx{Elementary row operation}s on either a \idx{system} of \idx{linear equation}s or on its corresponding \idx{augmented matrix} do not change the solutions (\cref{thm:erowop}):
\begin{itemize}
\item interchange two equations\slash rows; or
\item multiply an equation\slash row by a nonzero constant; or
\item add a multiple of an equation\slash row to another.
\end{itemize}

\item A system of \idx{linear equation}s or (augmented) matrix is in \bfidx{reduced row echelon form} (\rref) when (\cref{def:rref}) all the following hold:
  \begin{itemize}
\item any equations with all zero coefficients, or rows of the matrix consisting entirely of zeros, are at the bottom; 
\item in each nonzero equation\slash row, the first nonzero coefficient\slash entry is a one (called the \bfidx{leading one}), and is in a variable\slash column to the left of any \idx{leading one}s below it;
and
\item each variable\slash column containing a leading one has zero coefficients\slash entries in every other equation\slash row.\end{itemize}
A \bfidx{free variable} is any variable which is \emph{not} multiplied by a \idx{leading one} in the algebraic equations.

\itemme \idx{Gauss--Jordan elimination} solves systems by hand (\cref{pro:gje}):
\begin{enumerate}
\item Write down either the full symbolic form of the \idx{system} of \idx{linear equation}s, or the \idx{augmented matrix} of the system of \idx{linear equation}s.
\item Use \idx{elementary row operation}s to reduce the \idx{system}\slash{augmented matrix} to \idx{reduced row echelon form}.
\item If the resulting system is \idx{consistent}, then solve for the leading variables in terms of any remaining \idx{free variable}s.
\end{enumerate}

\itemhi For every system of linear equations \(A\xv=\bv\)\,, exactly one of the following holds (\cref{thm:fred}): there is either \idx{no solution}, or a \idx{unique solution}, or \idx{infinitely many solutions}.

\item A system of \idx{linear equation}s is called \bfidx{homogeneous} when the system may be written \(A\xv=\ov\) (\cref{def:homosys}),
otherwise the system is termed \bfidx{non-homogeneous}.

\itemme If \(A\xv=\ov\) is a \idx{homogeneous} system of \(m\)~{linear equation}s with \(n\)~variables where \(m<n\)\,, then the system has \idx{infinitely many solutions} (\cref{thm:feweqns}).



\subsubsection{Linear combinations span sets}

\itemhi A vector~\vv\ is a \bfidx{linear combination} of vectors \hlist\vv k\ if there are scalars \hlist ck\ (called the \bfidx{coefficient}s) such that \(\vv=\lincomb c\vv k\) (\cref{def:lincom}).

\item A system of \idx{linear equation}s \(A\xv=\bv\) is \bfidx{consistent} if and only if the right-hand side vector~\bv\ is a \idx{linear combination} of the columns of~\(A\) (\cref{thm:conlincom}).

\itemhi For any given set of vectors \(S=\{\hlist\vv k\}\), the set of all \idx{linear combination}s of \hlist\vv k\ is called the \bfidx{span} of \hlist\vv k, and is denoted by \(\Span\{\hlist\vv k\}\) or \(\Span S\) (\cref{def:span}).

\end{itemize}


\makeanswers
