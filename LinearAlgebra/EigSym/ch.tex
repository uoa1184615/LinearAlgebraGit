%!TEX root = ../larxxia.tex

\chapter{Eigenvalues and eigenvectors of symmetric matrices}
\label{ch:eesm}

\minitoc

\index{symmetric matrix|(}

Recall (\autoref{sec:amwm}) that a \idx{symmetric matrix}~\(A\) is a square matrix such that \(\tr A=A\)\,, that is, \(a_{ij}=a_{ji}\)\,.
For example, of the following two matrices, the first is symmetric, but the second is not:
\begin{equation*}
\begin{bmatrix} -2&4&0
\\4&2&-3
\\0&-3&1 \end{bmatrix};\qquad
\begin{bmatrix} -1&3&0
\\1&1&0
\\0&-3&1 \end{bmatrix}.
\end{equation*}

\begin{example} \label{eg:symsigns}
Compute some \svd{}s of random symmetric matrices, \(A=\usv\), observe in the \svd{}s that the columns of~\(U\) are always \(\pm\)~the columns of~\(V\) (well, almost always).
\begin{solution} 
Repeat as often us you like for any size of square matrix that you like  (one example is recorded here to two decimal places).  
\begin{enumerate}
\item Generate in \script\ some random symmetric matrix by adding a 
\idx{random matrix} to its transpose with \index{randn()@\texttt{randn()}}\verb|A=randn(5); A=A+A'| (\autoref{tbl:mtlbops}):
\setbox\ajrqrbox\hbox{\qrcode{% svd symmetric
A=randn(5); A=A+A'
[U,S,V]=svd(A)
}}%
\marginajrbox%
\begin{verbatim}
A =
  -0.45  -0.18   1.59  -0.96  -0.54
  -0.18  -0.24  -1.04   0.14   0.80
   1.59  -1.04  -2.87  -0.40   1.11
  -0.96   0.14  -0.40  -0.26  -1.90
  -0.54   0.80   1.11  -1.90   1.64
\end{verbatim}
This matrix is symmetric as \(a_{ij}=a_{ji}\)\,.
\item Find an \svd\ via \verb|[U,S,V]=svd(A)|
\begin{verbatim}
U =
  -0.41  -0.09  -0.28  -0.67   0.55
   0.25  -0.11  -0.05   0.53   0.80
   0.82  -0.19  -0.40  -0.36  -0.07
  -0.15   0.51  -0.80   0.27  -0.11
  -0.27  -0.83  -0.36   0.25  -0.22
S =
   4.28      0      0      0      0
      0   3.12      0      0      0
      0      0   1.65      0      0
      0      0      0   1.14      0
      0      0      0      0   0.51
V =
   0.41  -0.09   0.28  -0.67  -0.55
  -0.25  -0.11   0.05   0.53  -0.80
  -0.82  -0.19   0.40  -0.36   0.07
   0.15   0.51   0.80   0.27   0.11
   0.27  -0.83   0.36   0.25   0.22
\end{verbatim}
Observe the second and fourth columns of~\(U\) and~\(V\) are identical, and the other pairs of columns of~\(U\) and~\(V\) have opposite signs.
\end{enumerate}
Repeat for different random symmetric matrices and observe \(\uv_j=\pm \vv_j\) for every column~\(j\) (almost always).
\end{solution}
\end{example}

\index{symmetric matrix}Why, for symmetric matrices, are the columns of~\(U\) (almost) always \(\pm\)~the columns of~\(V\)?
The answer is connected to the following rearrangement of an \svd.  
Because \(A=\usv\), post-multiplying by~\(V\) gives \(AV=US\tr VV=US\)\,, and then the \(j\)th~column of the two sides of \(AV=US\) determines \(A\vv_j=\sigma_j\uv_j\)\,.
\autoref{eg:symsigns} indicates that for symmetric matrices~\(A\) we  find \(\uv_j=\pm\vv_j\) (almost always) so this last equation becomes \(A\vv_j=(\pm \sigma_j)\vv_j\).
This equation is of the important form \(A\vv=\lambda\vv\)\,.
\begin{aside}
The symbol~\bfidx{$\lambda$}\index{lambda, $\lambda$} is the Greek letter lambda, and denotes eigenvalues.
\end{aside}%
This form is important because it is the mathematical expression of the following geometric question: for what vectors~\vv\ does multiplication by~\(A\) just stretch\slash shrink~\vv\ by some \idx{scalar}~\(\lambda\)?

\begin{comment}
Dynamics of two masses on a spring (relate to molecular vibrations), and seek solutions in~\(\cos(ft)\) for eigenvalue \(\lambda=f^2\).  
As done for three masses in \autoref{eg:eig3vib}---if not omitted.  
Say symmetry from equal and opposite, claim holds even for gigantic structures with millions of interacting components.
\end{comment}


\paragraph{Solid modelling} Lean with a hand on a table\slash wall: the force changes depending upon the orientation of the surface.  
Similarly inside any solid: the internal forces\({}=A\vv\) where \(\vv\)~is the orthogonal \idx{unit vector} to the internal `surface'.  
In this solid-force scenario, the matrix~\(A\) is always symmetric.  
To know whether a material will break apart under pulling, or to crumble under compression, we need to know where the extreme forces are.  
The extreme forces are found as solutions to \(A\vv=\lambda\vv\) where \vv~gives the direction and \(\lambda\)~the strength of the extreme force.
To understand the potential failure of the material we thus need to solve equations in the \text{form \(A\vv=\lambda\vv\)\,.}




\endinput

