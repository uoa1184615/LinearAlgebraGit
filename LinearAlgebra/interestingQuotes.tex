
\section*{Quotes of interest}



\begin{quoted}{\parbox[t]{0.5\linewidth}{Manya Raman Sundstr\"om, 2015, New Scientist, \#3006}}
Those [proofs] found to be beautiful seem to give a more immediate sense of why the claim is true.  
For instance, a geometric proof of the relationship between the sides of a right-angled triangle, that compared areas of small triangles inside it, was considered more aesthetically pleasing than an algebraic proof.
This is probably because the algebraic proof gives no immediate sense of why the theorem is true.

\ldots\ 
Once you realise that mathematics is, in addition to its scientific merits, an essentially aesthetic subject, you realise that teaching it to the students without conveying its beauty might be to miss the essence, the very life, of the subject.
\end{quoted}




\begin{quoted}{\cite{Brunton2014}}
POD is sometimes referred to as \idx{principal component analysis} [38], 
the \idx{Karhunen--Lo\'eve decomposition}, \idx{empirical orthogonal functions} [33], or the \idx{Hotelling transform} [27].
\end{quoted}




\begin{quoted}{\idx{Wikipedia} (2015)}
The Schmidt decomposition is essentially a restatement of the singular value decomposition in a different context. 
\end{quoted}





\begin{quoted}{\cite[p.30]{Turner2014}}
The mathematics community is largely unaware of how math is used in other quantitative disciplines. 
The math curriculum has not changed much since the 1960s, and yet other related disciplines have changed substantially, and so we are really out of touch (speaking broadly not individually).
As an example, the singular-value decomposition (SVD) in linear algebra is a widely used technique in statistics, computer science, engineering, finance, and economics, and yet many pure mathematicians are unfamiliar with the topic, in large part because good numerical algorithms weren't developed until the 1960s and 1970s. 
To many mathematicians, linear algebra is the study of the algebraic properties of vector spaces and linear transformations. 
Some mathematicians pay little attention to the geometric and operator-theoretic properties of the field where applications are most prevalent. 
As so many other disciplines use the SVD, it is not only important that mathematicians understand what it is, but also teach it thoroughly in linear algebra and matrix analysis courses.
\end{quoted}



\begin{quoted}{\cite[p.39]{Halpern2003}}
Learning and recall are thus enhanced when learners integrate information from both verbal and visuospatial representations. 
\end{quoted}






\begin{quoted}{\cite[p.10]{CUPMguide2015}}
Cognitive Recommendation 1: 
Students should develop effective thinking and communication skills.
Major programs should include activities designed to promote students� progress in learning to \ldots\ 
use and compare analytical, visual, and numerical perspectives in exploring mathematics; \ldots

Cognitive Recommendation 2: 
Students should learn to link applications and theory. \ldots

Cognitive Recommendation 3:
Students should learn to use technological tools.
\end{quoted}






\begin{quoted}{\cite[p.39]{CUPMguide2015}}
Every Linear Algebra course should incorporate interesting applications, 
\end{quoted}






\begin{quoted}{\parbox{0.5\linewidth}{Mark Harris and Steve Kahn, New Scientist, 29 Aug 2015}}
The wide angle \textsc{lsst} [Large Synoptic Survey Telescope] will map the entire southern hemisphere every few nights.  \ldots\
How much data will it generate?  Fifteen terabytes every night!  
Over a decade the \textsc{lsst} will generate hundreds of petabytes of data, far more than everything ever written in history.  \ldots\
So how will you cope with the data deluge? 
The big challenge is not acquiring and storing the data but how you find anything in a database that big. \ldots\ 
The best contributions probably won't come from astronomers but from mathematicians, computer scientists and statisticians.
\end{quoted}


\begin{quoted}{Isaac Newton}
I keep the subject constantly before me and wait 'till the first dawnings open slowly, by little and little, into a full and clear light.
\end{quoted}



\begin{quoted}{\cite{Strogatz2012} [p.5]}
in mathematics our freedom lies in the questions we ask---and in how we pursue them---but not in the answers awaiting us.
\end{quoted}




\begin{quoted}{\cite{Strogatz2012} [p.237]}
in private, maths is occasionally insecure.
It has doubts.
It questions itself and isn't always sure it's right.
Especially where \idx{infinity} is concerned.
Infinity can can keep maths up at night, worrying, fidgeting, feeling existential dread. 
For there have been times in the history of maths when unleashing infinity wrought such mayhem, there were fears it might blow up the whole enterprise.
And that would be bad for business.
\end{quoted}



