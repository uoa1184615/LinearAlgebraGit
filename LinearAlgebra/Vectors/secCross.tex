%!TEX root = ../larxxia.tex
\section{The cross product}
\label{sec:cp}
%\secttoc

\index{cross product|(}


The dot product of \cref{sec:dpdal} is not the only way to multiply vectors.
In the three dimensions of the world we live in there is a second way to multiply vectors, called the cross product.
But for more than three dimensions, qualitatively different techniques are developed in \text{subsequent chapters.}

This section is optional for us, 
but is vital in many topics of science and engineering.



\needlines{13}
\subsubsection{Area of a parallelogram}
\index{parallelogram area|(}

\begin{wrapfigure}{r}{0pt} 
\begin{tikzpicture} 
\def\a{2} \def\b{0.5} \def\ab{2.5}
\def\c{1} \def\d{1.5} \def\cd{2.5}
\begin{axis}[axis equal image, axis lines=middle
    ,xtick={\a,\b,\ab},xticklabels={$v_1$,$w_1$,$v_1+w_1$}
    ,ytick={\c,\d,\cd},yticklabels={$v_2$,$w_2$,$v_2+w_2$}
    ,xmax=3,ymax=2.8,small
    ]
    \addplot[quiver={u=\a,v=\c},blue,-stealth] 
    coordinates {(0,0)(\b,\d)};
    \node[left] at (axis cs:\a,\c) {\small$(v_1,v_2)$};
    \addplot[quiver={u=\b,v=\d},blue,-stealth] 
    coordinates {(0,0)(\a,\c)};
    \node[right] at (axis cs:\b,\d) {\small$(w_1,w_2)$};
    \node[above] at (axis cs:\ab,\cd) {\small$(v_1+w_1,v_2+w_2) \hspace*{3.5em}$};
    \addplot[brown] coordinates {(\ab,0)(\ab,\cd)(0,\cd)};
    \addplot[brown] coordinates {(\ab,0)(\ab,\c)(\a,\c)(\a,0)};
    \addplot[brown] coordinates {(0,\cd)(\b,\cd)(\b,\d)(0,\d)};
\end{axis}
\end{tikzpicture}
\end{wrapfigure}
Consider the parallelogram drawn in blue.
It has sides given by vectors \(\vv=(v_1,v_2)\) and \(\wv=(w_1,w_2)\), as shown.
Let's determine the \idx{area} of the parallelogram. 
Its area is the containing rectangle less the two small rectangles and the four small triangles.
The two small rectangles have the same area, namely~\(w_1v_2\).
The two small triangles on the left and the right also have the same area, namely~\(\frac12w_1w_2\).
The two small triangles on the top and the bottom similarly have the same area, namely~\(\frac12v_1v_2\).
Thus, the parallelogram has 
\begin{eqnarray*}
\text{area}&=&(v_1+w_1)(v_2+w_2)-2w_1v_2-2\cdot\frac12w_1w_2-2\cdot\frac12v_1v_2
\nonumber
\\&=&v_1v_2+v_1w_2+w_1v_2+w_1w_2-2w_1v_2-w_1w_2-v_1v_2
\nonumber
\\&=&v_1w_2-v_2w_1\,. %\label{eq:cppara}
\end{eqnarray*}
In application, sometimes this right-hand side expression is negative because vectors~\vv\ and~\wv\ are the `wrong way' around.
Thus in general the \idx{parallelogram area}\({}=|v_1w_2-v_2w_1|\).


\begin{wrapfigure}[6]r{0pt}
\begin{tikzpicture} 
\def\a{3} \def\b{-1} 
\def\c{2} \def\d{4} 
\begin{axis}[footnotesize,font=\footnotesize,axis equal image, axis lines=middle
    ,xmin=-2.9,xmax=3.5,ymax=6.5
    ]
    \addplot[quiver={u=\a,v=\c},blue,-stealth] 
    coordinates {(0,0)(\b,\d)};
    \node[left] at (axis cs:\a,\c) {$(\a,\c)$};
    \addplot[quiver={u=\b,v=\d},blue,-stealth] 
    coordinates {(0,0)(\a,\c)};
    \node[above] at (axis cs:\b,\d) {$(\b,\d)\qquad$};
\end{axis}
\end{tikzpicture}\end{wrapfigure}
\begin{example} 
What is the \idx{area} of the parallelogram (illustrated to the right) whose edges are formed by the vectors~\((3,2)\) and~\((-1,4)\)?
\begin{solution} 
The parallelogram area\({}=|3\cdot4-2\cdot(-1)|=|12+2|=14\)\,.  
The illustration indicates that this area must be about right, as with imagination one could cut the area and move the parts about to form a rectangle roughly~\(3\) by~\(5\), and hence the area should be roughly~\(15\).
\end{solution}
\end{example}


\begingroup
\def\temp{\begin{tikzpicture} 
\def\a{5} \def\b{2} \def\ab{7}
\def\c{3} \def\d{-2} \def\cd{1}
\begin{axis}[footnotesize,font=\footnotesize
    ,axis equal image, axis lines=middle
    ,ymin=-2.5,ymax=3.5,xmax=7.5
    ]
    \addplot[quiver={u=\a,v=\c},blue,-stealth] 
    coordinates {(0,0)(\b,\d)};
    \node[left] at (axis cs:\a,\c) {$(\a,\c)$};
    \addplot[quiver={u=\b,v=\d},blue,-stealth] 
    coordinates {(0,0)(\a,\c)};
    \node[right] at (axis cs:\b,\d) {$(\b,\d)$};
\end{axis}
\end{tikzpicture}}
\begin{activity}[\temp] 
What is the \idx{area} of the parallelogram (illustrated to the right) whose edges are formed by the vectors~\((5,3)\) and~\((2,-2)\)?
\actposs[4]{\(16\)}{\(4\)}{\(11\)}{\(19\)}
\vspace{1ex}
\end{activity}
\endgroup

Interestingly, we meet this expression for \idx{area}, \(v_1w_2-v_2w_1\), in another context: that of equations for a plane and its \text{normal vector.}

\index{parallelogram area|)}





\subsubsection{Normal vector to a plane}
Recall \cref{sec:nvep} introduced that we describe planes either via an equation such as \(x-y+3z=6\) or via a parametric description such as \(\xv=(1,1,2)+(1,1,0)s+(0,3,1)t\)\,.
These determine the same plane; they are just different algebraic descriptions.
One converts between these two descriptions using the \text{cross product.}




\begin{example} \label{eg:nviax}
Derive that the plane described parametrically by \(\xv=(1,1,2)+(1,1,0)s+(0,3,1)t\) has normal equation \(x-y+3z=6\)\,.
\begin{solution} 
The key to deriving the normal equation is to find that a \idx{normal vector} to the plane is~\((1,-1,3)\).
This normal vector comes from the two vectors that multiply the parameters in the parametric form, \((1,1,0)\) and~\((0,3,1)\).
The following mysterious looking procedure may be a convenient way for you to remember an otherwise involved formula: if you prefer to remember the formula of \cref{def:cp}, then use that instead.
(Those who have computed \(3\times3\) determinants may recognize that the following has the same pattern---see \cref{ch:ddm}.)
Write the vectors as two consecutive columns, following a first column of the \emph{symbols} of the \idx{standard unit vector}s~\iv, \jv, and~\kv, in
\setlength{\unitlength}{1.2ex}
\def\abc#1{\begin{vmatrix}\begin{picture}(5.3,6)
%\put(0,0){\framebox(5,5){}}
\put(0,4){$\iv$}\put(2,4){$1$}\put(4,4){$0$}
\put(0,2){$\jv$}\put(2,2){$1$}\put(4,2){$3$}
\put(0,0){$\kv$}\put(2,0){$0$}\put(4,0){$1$}
\ifnum1=#1\put(0.5,-0.5){\line(0,1)6}\put(-0.5,4.5){\line(1,0)6}\fi
\ifnum2=#1\put(0.5,-0.5){\line(0,1)6}\put(-0.5,2.5){\line(1,0)6}\fi
\ifnum3=#1\put(0.5,-0.5){\line(0,1)6}\put(-0.5,0.5){\line(1,0)6}\fi
\end{picture}\end{vmatrix}}
\def\ab#1#2#3#4{\begin{vmatrix}\begin{picture}(3,4)
\put(0,2){$#1$}\put(2,2){$#2$}
\put(0,0){$#3$}\put(2,0){$#4$}
\color{red}\put(-0.5,-0.5){\line(1,1)4}
\color{blue}\put(-0.5,3.5){\line(1,-1)4}
\end{picture}\end{vmatrix}}
\begin{eqnarray*}
\nv&=& \abc0 
\\&&\parbox{23em}{(then cross out 1st column and each row in turn, multiplying each by common entry, with alternating sign)}
\\&=&\iv\abc1-\jv\abc2+\kv\abc3
\\&=&\iv\begin{vmatrix} 1&3\\0&1 \end{vmatrix}
-\jv\begin{vmatrix} 1&0\\0&1 \end{vmatrix}
+\kv\begin{vmatrix} 1&0\\1&3 \end{vmatrix}
\\&&\parbox{20em}{(then draw diagonals, then subtract product of red diagonal from product of the blue)}
\\&=&\iv\ab1301
-\jv\ab1001
+\kv\ab1013
\\&=&\iv(1\cdot1-0\cdot3)
-\jv(1\cdot1-0\cdot0)
+\kv(1\cdot3-1\cdot0)
\\&=&\iv-\jv+3\kv\,.
\end{eqnarray*}
Using this normal vector, the equation of the plane must be of the form \(x-y+3z={}\)constant.
Since the plane goes through point~\((1,1,2)\), the constant\({}=1-1+3\cdot2=6\); that is, the plane is \(x-y+3z=6\) (as given).
\end{solution}
\end{example}




\begin{activity}  
Use the procedure of \cref{eg:nviax} to derive a \idx{normal vector} to the plane described in parametric form as \(\xv=(4,-1,-2)+(1,-2,1)s+(2,-3,-2)t\).  
Which of the following is your computed normal vector?
%pqr=0+round(randn(3)*2),n=cross(pqr(2,:),pqr(3,:))
\actposs[4]{\((7,4,1)\)}{\((5,6,7)\)}{\((-4,4,-10)\)}{\((2,-2,5)\)}
\end{activity}




\subsubsection{Definition of a cross product}

\paragraph{General formula}
The procedure used in \cref{eg:nviax} to derive a \idx{normal vector} leads to an algebraic formula.  
Let's apply the same procedure to two general vectors \(\vv=(v_1,v_2,v_3)\) and \(\wv=(w_1,w_2,w_3)\).
The procedure computes
{%%%%%%%%%%%%%%%%%%
\setlength{\unitlength}{1.3ex}
\def\abc#1{\begin{vmatrix}\begin{picture}(6.3,6)
%\put(0,0){\framebox(5,5){}}
\put(0,4){$\iv$}\put(2,4){$v_1$}\put(4,4){$w_1$}
\put(0,2){$\jv$}\put(2,2){$v_2$}\put(4,2){$w_2$}
\put(0,0){$\kv$}\put(2,0){$v_3$}\put(4,0){$w_3$}
\ifnum1=#1\put(0.5,-0.5){\line(0,1)6}\put(-0.5,4.5){\line(1,0)6}\fi
\ifnum2=#1\put(0.5,-0.5){\line(0,1)6}\put(-0.5,2.5){\line(1,0)6}\fi
\ifnum3=#1\put(0.5,-0.5){\line(0,1)6}\put(-0.5,0.5){\line(1,0)6}\fi
\end{picture}\end{vmatrix}}
\def\ab#1#2{\begin{vmatrix}\begin{picture}(4,4)
\put(0,2){$v_#1$}\put(2,2){$w_#1$}
\put(0,0){$v_#2$}\put(2,0){$w_#2$}
\color{red}\put(-0.5,-0.5){\line(1,1)4}
\color{blue}\put(-0.5,3.5){\line(1,-1)4}
\end{picture}\end{vmatrix}}
\begin{eqnarray*}
\nv&=& \abc0 
\\&&\parbox{23em}{(then cross out 1st column and each row in turn, multiplying each by common entry, with alternating sign)}
\\&=&\iv\abc1-\jv\abc2+\kv\abc3
\\&=&\iv\begin{vmatrix} v_2&w_2\\v_3&w_3 \end{vmatrix}
-\jv\begin{vmatrix} v_1&w_1\\v_3&w_3 \end{vmatrix}
+\kv\begin{vmatrix} v_1&w_1\\v_2&w_2 \end{vmatrix}
\\&&\parbox{20em}{(then draw diagonals, then subtract product of red diagonal from product of the blue)}
\\&=&\iv\ab23
-\jv\ab13
+\kv\ab12
\\&=&\iv(v_2w_3-v_3w_2)
-\jv(v_1w_3-v_3w_1)
+\kv(v_1w_2-v_2w_1).
\end{eqnarray*}
}%%%%%%%%%%%%%%%%%%%%
We use this formula to define the cross product algebraically, and then see what it means geometrically.

\begin{definition} \label{def:cp}
Let \(\vv=(v_1,v_2,v_3)\) and \(\wv=(w_1,w_2,w_3)\) be two vectors in~\(\RR^3\).
The \bfidx{cross product}  (or \bfidx{vector product}) \(\vv\times\wv\) is defined algebraically as
\index{i@$\iv$}\index{j@$\jv$}\index{k@$\kv$}%
\begin{equation*}
\vv\times\wv:=\iv(v_2w_3-v_3w_2)
+\jv(v_3w_1-v_1w_3)
+\kv(v_1w_2-v_2w_1).
\end{equation*}
\end{definition}


\needlines7
\begin{example} \label{eg:cpijk}
Among the \idx{standard unit vector}s, derive that 
\index{i@$\iv$}\index{j@$\jv$}\index{k@$\kv$}%
\begin{Parts}
\item \(\iv\times\jv=\kv\)\,, \item \(\jv\times\iv=-\kv\)\,,
\item \(\jv\times\kv=\iv\)\,, \item \(\kv\times\jv=-\iv\)\,,
\item \(\kv\times\iv=\jv\)\,, \item \(\iv\times\kv=-\jv\)\,,
\item \(\iv\times\iv=\jv\times\jv=\kv\times\kv=\ov\)\,.
\end{Parts}
\begin{solution} Using \cref{def:cp}:
\begin{eqnarray*}
\iv\times\jv&=&(1,0,0)\times(0,1,0)
\\&=&\iv(0\cdot0-0\cdot1)
+\jv(0\cdot0-1\cdot0)
+\kv(1\cdot1-0\cdot0)
\\&=&\kv\,;
\\\jv\times\iv&=&(0,1,0)\times(1,0,0)
\\&=&\iv(1\cdot0-0\cdot0)
+\jv(0\cdot1-0\cdot0)
+\kv(0\cdot0-1\cdot1)
\\&=&-\kv\,;
\\\iv\times\iv&=&(1,0,0)\times(1,0,0)
\\&=&\iv(0\cdot0-0\cdot0)
+\jv(0\cdot1-1\cdot0)
+\kv(1\cdot0-0\cdot1)
\\&=&\ov\,.
\end{eqnarray*}
\cref{ex:cpijk} asks you to correspondingly establish the other six identities.
\end{solution}
\end{example}


The cross products of this \cref{eg:cpijk} most clearly demonstrate the orthogonality of a cross product to its two argument vectors (\cref{thm:cpga}), and that the direction is in the so-called \idx{right-hand sense} (\cref{thm:cpgb}).



\begin{activity} 
Use \cref{def:cp} to find the cross product of \((-4,1,-1)\) and \((-2,2,1)\) is which one of the following:
\actposs[4]{\((3,6,-6)\)}{\((-3,-6,6)\)}{\((3,-6,-6)\)}{\((-3,-6,6)\)}
\end{activity}




\subsubsection{Geometry of a cross product}


\begin{example}[\idx{parallelogram area}] \label{eg:cppara}
Let's revisit the introduction to this section.
Consider the parallelogram in the \(x_1x_2\)-plane with edges formed by the \(\RR^3\)~vectors \(\vv=(v_1,v_2,0)\) and \(\wv=(w_1,w_2,0)\).
At the start of this \cref{sec:cp} we derived that the parallelogram formed by these vectors has \idx{area}\({}=|v_1w_2-v_2w_1|\).
Compare this area with the cross product
\begin{eqnarray*}
\vv\times\wv&=&\iv(v_2\cdot0-0\cdot w_2)
+\jv(0\cdot w_1-v_1\cdot0)
+\kv(v_1w_2-v_2w_1)
\\&=&\iv0+\jv0+\kv(v_1w_2-v_2w_1)
\\&=&\kv(v_1w_2-v_2w_1).
\end{eqnarray*}
Consequently, the \idx{length} of this cross product equals the area of the parallelogram formed by~\vv\ and~\wv\ (\cref{thm:cpgd}).
(Also the direction of the cross product,~\(\pm\kv\), is orthogonal to the \(x_1x_2\)-plane containing the two vectors---\cref{thm:cpga}).
\end{example}


\begingroup
\def\temp{\qview{30}{35}{\begin{tikzpicture} 
\begin{axis}[footnotesize,font=\footnotesize,axis equal,view={\q}{30}
    ,xlabel={$x_1$},ylabel={$x_2$},zlabel={$x_3$},label shift={-1.5ex}
    ]
    \threev[above]20{0.5}{\vec v};
    \threev[above]102{\vec w};
\end{axis}
\end{tikzpicture}}}
\begin{activity}[\temp]
Using property \ref{thm:cpgb} of the next theorem, in which direction is the cross product \(\vv\times\wv\) for the two vectors illustrated in stereo to the right?
\actposs{\(-\jv\)}{\(+\iv\)}{\(+\jv\)}{\(-\iv\)}
\end{activity}
\endgroup



\begin{figbox}{\begin{tikzpicture}
  \begin{axis}[small,thick, axis lines=none,ymax=1.3,ymin=-0.3,xmin=-0.3,xmax=1.3]
    \addplot graphics [xmin=0,xmax=1,ymin=0,ymax=1]
      {Vectors/right-hand-rule.jpg};
      % this is AJRs photo of AJRs hand
    \addplot[blue,quiver={u=-0.05,v=0.5,scale arrows=1.07},-stealth] coordinates {(0.55,0.55)};
    \node[above] at (axis cs:0.5,1.05) {$\vv$};
    \addplot[blue,quiver={u=-0.6,v=0.2,scale arrows=1.07},-stealth] coordinates {(0.55,0.55)};
    \node[left] at (axis cs:-0.05,0.77) {$\wv$};
    \addplot[blue,quiver={u=-0.4,v=-0.6,scale arrows=1.07},-stealth] coordinates {(0.55,0.55)};
    \node[below] at (axis cs:0.15,-0.05) {$\vv\times\wv$};
  \end{axis}
\end{tikzpicture}}%
\begin{theorem}[cross product geometry] \label{thm:cpg}
Let \vv\ and~\wv\ be two vectors in~\(\RR^3\):
\begin{enumerate}[ref=\ref{thm:cpg}(\alph*)]
\item\label[theorem]{thm:cpga} the vector~\(\vv\times\wv\) is \idx{orthogonal} to both~\vv\ and~\wv;

\item\label[theorem]{thm:cpgb} 
the \index{cross product direction}direction of~\(\vv\times\wv\) is in the \bfidx{right-hand sense},
in that if \vv~is in the direction of your thumb, and \wv~is in the direction of your straight index finger, then \(\vv\times\wv\) is in the direction of your bent second\slash longest finger---all on your right hand as illustrated to \text{the right;} 

\item\label[theorem]{thm:cpgc} \(|\vv\times\wv|=|\vv|\,|\wv|\sin\theta\) where \(\theta\)~is the \idx{angle} between vectors~\vv\ and~\wv\ (\(0\leq\theta\leq\pi\), equivalently \(0^\circ\leq\theta\leq180^\circ\)); and

\item\label[theorem]{thm:cpgd} the \idx{length}~\(|\vv\times\wv|\) is the \idx{area} of the parallelogram\index{parallelogram area} with edges~\vv\ and~\wv.
\end{enumerate}
\end{theorem}
\end{figbox}



\begin{proof} 
Let \(\vv=(v_1,v_2,v_3)\) and \(\wv=(w_1,w_2,w_3)\).

\begin{description}
\item[\ref{thm:cpga}] Recall that two vectors are orthogonal if their dot product is zero (\cref{def:orthovec}).
To determine orthogonality between~\vv\ and the cross product \(\vv\times\wv\), consider
\begin{eqnarray*}
\vv\cdot(\vv\times\wv)
&=&(v_1\iv+v_2\jv+v_3\kv)
\cdot\big[\iv(v_2w_3-v_3w_2)
\\&&{}
+\jv(v_3w_1-v_1w_3)
+\kv(v_1w_2-v_2w_1)\big]
\\&=&v_1(v_2w_3-v_3w_2)
+v_2(v_3w_1-v_1w_3)
\\&&{}
+v_3(v_1w_2-v_2w_1)
\\&=&v_1v_2w_3-v_1v_3w_2
+v_2v_3w_1
\\&&{}
-v_1v_2w_3
+v_1v_3w_2-v_2v_3w_1
\quad{}=0
\end{eqnarray*}
as each term in the penultimate line cancels with the term underneath in the last line.
Since the dot product is zero, the cross product \(\vv\times\wv\) is orthogonal to vector~\vv.

Similarly, \(\vv\times\wv\) is orthogonal to~\wv\  (\cref{ex:cpga}).

\item[\ref{thm:cpgb}] This right-handed property follows from the convention that the \idx{standard unit vector}s~\iv, \jv, and~\kv\ are right-handed: that if \iv~is in the direction of your thumb, and \jv~is in the direction of your straight index finger, then \kv~is in the direction of your bent second\slash longest finger---all on your right hand.

We prove only for the case of vectors in the \(x_1x_2\)-plane, in which case \(\vv=(v_1,v_2,0)\) and \(\wv=(w_1,w_2,0)\), and when both \(v_1,w_1>0\)\,.
One example is in stereo below.
\begin{center}
\qview{30}{35}{\begin{tikzpicture} 
\def\a{1.5} \def\b{0.5} 
\def\c{1} \def\d{2} 
\begin{axis}[footnotesize,font=\footnotesize,axis equal,view={\q}{30}
    ,xmin=-0.4,xmax=2.4,ymin=-0.4,ymax=2.4,xtick={0,1,2},ztick={0,1}
    ,xlabel={$x_1$},ylabel={$x_2$},zlabel={$x_3$},label shift={-1.5ex}
    ]
    \addplot3[quiver={u=\a,v=\c,w=0},blue,-stealth,thick] 
    coordinates {(0,0,0)};
    \node[right] at (axis cs:\a,\c,0) {$\vv$};
    \addplot3[quiver={u=\b,v=\d,w=0},blue,-stealth,thick] 
    coordinates {(0,0,0)};
    \node[above] at (axis cs:\b,\d,0) {$\wv$};
    \addplot3[quiver={u=0,v=0,w=1},red,-stealth,thick] 
    coordinates {(0,0,0)};
    \node[above] at (axis cs:0,0,1) {$+\kv$};
\end{axis}
\end{tikzpicture}}
\end{center}
\cref{eg:cppara} derived the cross product \(\vv\times\wv=\kv(v_1w_2-v_2w_1)\).
Consequently, this cross product is in the~\(+\kv\) direction only when \(v_1w_2-v_2w_1>0\) (it is in the~\(-\kv\) direction in the complementary case when \(v_1w_2-v_2w_1<0\)). 
This inequality for~\(+\kv\) rearranges to \(v_1w_2>v_2w_1\)\,.
Dividing by the positive~\(v_1w_1\) requires \(\frac{w_2}{w_1}>\frac{v_2}{v_1}\)\,.
That is, in the \(x_1x_2\)-plane the `slope' of vector~\wv\ must be greater than the `slope' of vector~\vv.
In this case, if \vv~is in the direction of your thumb on your right hand, and \wv~is in the direction of your straight index finger, then your bent second\slash longest finger is in the direction~\(+\kv\) as required by the cross product \(\vv\times\wv\)\,.

\item[\ref{thm:cpgc}] \cref{ex:cpgc} establishes the identity \(|\vv\times\wv|^2=|\vv|^2|\wv|^2-(\vv\cdot\wv)^2\).
From \cref{thm:anglev} substitute \(\vv\cdot\wv=|\vv||\wv|\cos\theta\) into this identity:
\begin{eqnarray*}
|\vv\times\wv|^2&=&|\vv|^2|\wv|^2-(\vv\cdot\wv)^2
\\&=&|\vv|^2|\wv|^2-(|\vv||\wv|\cos\theta)^2
\\&=&|\vv|^2|\wv|^2-|\vv|^2|\wv|^2\cos^2\theta
\\&=&|\vv|^2|\wv|^2(1-\cos^2\theta)
\\&=&|\vv|^2|\wv|^2\sin^2\theta\,.
\end{eqnarray*}
Take the square-root of both sides to determine \(|\vv\times\wv|=\pm|\vv||\wv|\sin\theta\)\,.
But \(\sin\theta\geq0\) since the angle \(0\leq\theta\leq\pi\)\,, and all the lengths are also\({}\geq0\)\,, so only the plus case applies.
That is, the length \(|\vv\times\wv|=|\vv||\wv|\sin\theta\) \text{as required.}

\item[\ref{thm:cpgd}] \ \\
\begin{figbox}{\rotatebox{20}{\begin{tikzpicture} 
\def\a{1} \def\b{0.7} 
\def\c{0} \def\d{1.5} 
\begin{axis}[axis equal image, axis lines=middle,xtick={0},ytick={0}
    ,xmax=1.8,ymax=1.9,ymin=-0.05,xmin=-0.05,footnotesize,font=\small
    ,xlabel={base},ylabel={height}
    ]
    \addplot[quiver={u=\a,v=\c},blue,-stealth,thick] 
    coordinates {(0,0)(\b,\d)};
    \node[above] at (axis cs:\a,\c) {$\vv\quad$};
    \addplot[quiver={u=\b,v=\d},blue,-stealth,thick] 
    coordinates {(0,0)(\a,\c)};
    \node[left] at (axis cs:\b,\d) {$\wv$};
    \node[above] at (axis cs:0,0) {$\qquad\theta$};
\end{axis}
\end{tikzpicture}}}%
Consider the plane containing the vectors~\vv\ and~\wv, 
and hence containing the parallelogram formed by these vectors---as illustrated to the right.
Using vector~\vv\ as the base of the parallelogram, with length~\(|\vv|\), by basic trigonometry the height of the parallelogram is then \(|\wv|\sin\theta\).
Hence the area of the parallelogram is the product 
\(\text{base}\cdot\text{height}=|\vv||\wv|\sin\theta=|\vv\times\wv|\) by the previous part~\ref{thm:cpgc}.\end{figbox}
\end{description}
\end{proof}

\needlines7
\begin{wrapfigure}[7]r{0pt}
\qview{30}{34}{\begin{tikzpicture} 
\def\a{-2} \def\b{2} 
\def\c{0} \def\d{2} 
\def\e{1} \def\f{1}
\begin{axis}[footnotesize,font=\footnotesize,axis equal,view={\q}{25}
    ,xlabel={$x_1$},ylabel={$x_2$},zlabel={$x_3$},label shift={-1.5ex}
    ,ytick={0,2} ]
    \threev[above]{\a}{\c}{\e}{\vec v};
    \addplot3[quiver={u=\a,v=\c,w=\e},blue,-stealth,thick] 
    coordinates {(\b,\d,\f)};
    \threev[above]{\b}{\d}{\f}{\vec w};
    \addplot3[quiver={u=\b,v=\d,w=\f},blue,-stealth,thick] 
    coordinates {(\a,\c,\e)};
    \addplot3[gray] coordinates 
    {((\a)+(\b),(\c)+(\d),0)((\a)+(\b),(\c)+(\d),\e+\f)};
\end{axis}
\end{tikzpicture}}
\end{wrapfigure}
\begin{example} \label{eg:apvw}
Find the \idx{area} of the parallelogram\index{parallelogram area} with edges formed by vectors
\(\vv=(-2,0,1)\) and \(\wv=(2,2,1)\)---as illustrated in stereo to the right.
\begin{solution} 
The area is the length of the cross product
\begin{eqnarray*}
\vv\times\wv
&=&\iv(0\cdot 1-1\cdot 2)
+\jv(1\cdot 2-(-2)\cdot 1)
+\kv((-2)\cdot 2-0\cdot 2)
\\&=&-2\iv+4\jv-4\kv\,.
\end{eqnarray*}
Then the parallelogram area \(|\vv\times\wv|=\sqrt{(-2)^2+4^2+(-4)^2}
=\sqrt{4+16+16}=\sqrt{36}=6\)\,.
\aqed\par
\end{solution}
\end{example}

\begingroup
\def\temp{\qview{30}{34}{\begin{tikzpicture} 
\def\a{-2} \def\b{2} 
\def\c{-1} \def\d{0} 
\def\e{0} \def\f{-1}
\begin{axis}[footnotesize,font=\footnotesize,axis equal,view={\q}{25}
    ,xlabel={$x_1$},ylabel={$x_2$},zlabel={$x_3$},label shift={-1.5ex}
    ,ytick={-2,0} ]
    \threev[above]{\a}{\c}{\e}{\vec v};
    \addplot3[quiver={u=\a,v=\c,w=\e},blue,-stealth,thick] 
    coordinates {(\b,\d,\f)};
    \threev[above]{\b}{\d}{\f}{\vec w};
    \addplot3[quiver={u=\b,v=\d,w=\f},blue,-stealth,thick] 
    coordinates {(\a,\c,\e)};
    \addplot3[gray] coordinates 
    {((\a)+(\b),(\c)+(\d),0)((\a)+(\b),(\c)+(\d),\e+\f)};
\end{axis}
\end{tikzpicture}}}
\begin{activity}[\temp]
% u=0+round(randn(1,3)*3), v=0+round(randn(1,3)*3), uv=cross(u,v), area=norm(uv)
What is the \idx{area} of the parallelogram\index{parallelogram area} (in stereo to the right) with edges formed by vectors
\(\vv=(-2,1,0)\) and \(\wv=(2,0,-1)\)?
\actposs{\(3\)}{\(1\)}{\(\sqrt{5}\)}{\(5\)}
\end{activity}
\endgroup




\needlines8
\begin{wrapfigure}r{0pt}
\qview{26}{31}{\begin{tikzpicture} 
\def\a{-2} \def\b{2} 
\def\c{3} \def\d{2} 
\def\e{2} \def\f{3}
\begin{axis}[footnotesize,font=\footnotesize,axis equal,view={\q}{25}
    ,xlabel={$x_1$},ylabel={$x_2$},zlabel={$x_3$},label shift={-1.5ex}
    ]
    \threev[above]{\a}{\c}{\e}{\vec v};
    \threev[above]{\b}{\d}{\f}{\vec w};
    \addplot3[quiver={u=1,v=2,w=-2},red,-stealth,thick] 
    coordinates {(0,0,0)};
    \node[below] at (axis cs:1,2,-2) {$\nv$};
\end{axis}
\end{tikzpicture}}
\end{wrapfigure}
\begin{example} \label{eg:cpnvp}
Find a \idx{normal vector} to the plane containing the two vectors \(\vv=-2\iv+3\jv+2\kv\) and \(\wv=2\iv+2\jv+3\kv\) ---illustrated to the right.
Hence find an \idx{equation of the plane} given parametrically as \(\xv=-2\iv-\jv+3\kv+(-2\iv+3\jv+2\kv)s+(2\iv+2\jv+3\kv)t\)\,.
\index{i@$\iv$}\index{j@$\jv$}\index{k@$\kv$}%

\begin{solution} 
Use \cref{def:cp} of the cross product to find a normal vector:
\begin{eqnarray*}
\vv\times\wv&=&
\iv(3\cdot 3-2\cdot 2)
+\jv(2\cdot 2-(-2)\cdot 3)
+\kv((-2)\cdot 2-3\cdot 2)
\\&=&5\iv+10\jv-10\kv\,.
\end{eqnarray*}
A normal vector is any vector proportional to this, so we could divide by five and choose normal vector \(\nv=\iv+2\jv-2\kv\) (as illustrated above).

An equation of the plane through \(-2\iv-\jv+3\kv\) is then given by the dot product
\begin{eqnarray*}
&&(\iv+2\jv-2\kv)\cdot[(x+2)\iv+(y+1)\jv+(z-3)\kv]=0\,,
\\\text{that is,}&& x+2+2y+2-2z+6=0\,,
\\\text{that is,}&& x+2y-2z+10=0
\end{eqnarray*}
is the required normal equation of the plane.
\aqed
\end{solution}
\end{example}




\subsubsection{Algebraic properties of a cross product}

\cref{ex:cppa,ex:cppb,ex:cppc} establish three of the following four useful algebraic properties of the cross product.

\begin{theorem}[cross product properties] \label{thm:cpp}
Let \uv, \vv, and~\wv\ be vectors in~\(\RR^3\), and \(c\)~be a \idx{scalar}:
\begin{enumerate}[ref=\ref{thm:cpp}(\alph*)]
\item\label[theorem]{thm:cppz} \(\vv\times\vv=\ov\);
\item\label[theorem]{thm:cppa} \(\wv\times\vv=-(\vv\times\wv)\) \quad(not commutative);\index{commutative law}
\item\label[theorem]{thm:cppb} \((c\vv)\times\wv=c(\vv\times\wv)=\vv\times(c\wv)\);
\item\label[theorem]{thm:cppc} \(\uv\times(\vv+\wv)=\uv\times\vv+\uv\times\wv\) \quad(\idx{distributive law}).
\end{enumerate}
\end{theorem}


\begin{proof} 
Let's prove property~\ref{thm:cppz} two ways---algebraically and geometrically.  \cref{ex:cppa,ex:cppb,ex:cppc} ask you to prove the other properties.
\begin{itemize}
\item Algebraically:  with vector \(\vv=(v_1,v_2,v_3)\),  \cref{def:cp} gives
\begin{eqnarray*}
\vv\times\vv
&=&\iv(v_2v_3-v_3v_2)
+\jv(v_3v_1-v_1v_3)
+\kv(v_1v_2-v_2v_1)
\\&=&0\iv+0\jv+0\kv=\ov\,.
\end{eqnarray*}
\item Geometrically: 
%from \cref{thm:cpgc} \(|\vv\times\vv|=|\vv||\vv|\sin\theta\) where \(\theta\)~is the angle between~\vv\ and~\vv, and so \(\theta=0\)\,.
%Since \(\sin\theta=\sin0=0\)\,, the length \(|\vv\times\vv|=0\) and so \(\vv\time\vv=\ov\) (\cref{thm:veclen0}).
from \cref{thm:cpgd}, \(|\vv\times\vv|\) is the area of the parallelogram with edges~\vv\ and~\vv.
But such a parallelogram has zero area, so \(|\vv\times\vv|=0\)\,.
Since the only vector of length zero is the zero vector (\cref{thm:veclen0}), \(\vv\times\vv=\ov\).
\end{itemize}
\end{proof}





\begin{example} 
As an example of \cref{thm:cppa}, \cref{eg:cpijk} shows that \(\iv\times\jv=\kv\)\,, whereas reversing the order of the cross product gives the negative \(\jv\times\iv=-\kv\)\,.  
Given \cref{eg:cpnvp} derived \(\vv\times\wv=5\iv+10\jv-10\kv\) in the case when \(\vv=-2\iv+3\jv+2\kv\) and \(\wv=2\iv+2\jv+3\kv\)\,, what is \(\wv\times\vv\)?
\begin{solution} 
By \ref{thm:cppa},
\(\wv\times\vv=-(\vv\times\wv)=-5\iv-10\jv+10\kv\)\,.
\end{solution}
\end{example}


\begin{example} 
Given \((\iv+\jv+\kv)\times(-2\iv-\jv)=\iv-2\jv+\kv\)\,, what is \((3\iv+3\jv+3\kv)\times(-2\iv-\jv)\)?
\begin{solution} 
The first vector is \(3(\iv+\jv+\kv)\) so by \cref{thm:cppb},
\begin{align*}
&(3\iv+3\jv+3\kv)\times(-2\iv-\jv)
\\&=[3(\iv+\jv+\kv)]\times(-2\iv-\jv)
\\&=3[(\iv+\jv+\kv)\times(-2\iv-\jv)]
\\&=3[\iv-2\jv+\kv]=3\iv-6\jv+3\kv\,.
\end{align*}
\end{solution}
\end{example}


\begin{activity}  
For vectors \(\uv=-\iv+3\kv\)\,, \(\vv=\iv+3\jv+5\kv\)\,, and \(\wv=-2\iv+\jv-\kv\) you are given that 
\begin{eqnarray*}
&&\uv\times\vv=-9\iv+8\jv-3\kv\,,
\\&&\uv\times\wv=-3\iv-7\jv-\kv\,,
\\&&\vv\times\wv=-8\iv-9\jv+7\kv\,.
\end{eqnarray*}
Which is the cross product \((-\iv+3\kv)\times(-\iv+4\jv+4\kv)\)?
\actposs{\(-12\iv+\jv-4\kv\)}
{\(\iv-17\jv+10\kv\)}
{\(-11\iv-16\jv+6\kv\)}
{\(-17\iv-\jv+4\kv\)}
Also, which is \((\iv+3\jv+5\kv)\times(-3\iv+\jv+2\kv)\)? %\(\iv-17\jv+10\kv\)
\end{activity}




\begin{example} 
The properties of \cref{thm:cpp} empower algebraic manipulation.
Use such algebraic manipulation, and the identities among \idx{standard unit vector}s of \cref{eg:cpijk}, compute the cross product \((\iv-\jv)\times(4\iv+2\kv)\).
\begin{solution} In full detail:
\begin{align*}
&(\iv-\jv)\times(4\iv+2\kv)
\\&=(\iv-\jv)\times(4\iv)+(\iv-\jv)\times(2\kv) 
\quad(\text{by \ref{thm:cppc}})
\\&=4(\iv-\jv)\times\iv+2(\iv-\jv)\times\kv
\quad(\text{by \ref{thm:cppb}})
\\&=-4\iv\times(\iv-\jv)-2\kv\times(\iv-\jv)
\quad(\text{by \ref{thm:cppa}})
\\&=-4[\iv\times\iv+\iv\times(-\jv)]-2[\kv\times\iv+\kv\times(-\jv)]
\quad(\text{by \ref{thm:cppc}})
\\&=-4[\iv\times\iv-\iv\times\jv]-2[\kv\times\iv-\kv\times\jv]
\quad(\text{by \ref{thm:cppb}})
\\&=-4[\ov-\kv]-2[\jv-(-\iv)]
\quad(\text{by \cref{eg:cpijk}})
\\&=-2\iv-2\jv+4\kv\,.
\end{align*}
\end{solution}
\end{example}







\needlines7
\subsubsection{Volume of a parallelepiped}

\index{parallelepiped volume|(}
\newcommand{\pppd}[1]{% draw a parallelepiped
\qview{30}{34}{\begin{tikzpicture} 
\def\ua{0.5}\def\ub{0.5}\def\uc{1}
\def\va{2}\def\vb{0.5}\def\vc{0}
\def\wa{0.5}\def\wb{1.3}\def\wc{0}
\def\vw{1.7}
\begin{axis}[small,font=\footnotesize,axis equal,view={\q}{25}
    ,axis lines=none ]
    \addplot3[quiver={u=\ua,v=\ub,w=\uc},blue,-stealth,thick] 
    coordinates {(0,0,0)(\va,\vb,\vc)(\wa,\wb,\wc)(\va+\wa,\vb+\wb,\vc+\wc)};
    \node[below] at (axis cs:\ua,\ub,\uc) {$\vec u$};
    \addplot3[quiver={u=\va,v=\vb,w=\vc},blue,-stealth,thick] 
    coordinates {(0,0,0)(\ua,\ub,\uc)(\wa,\wb,\wc)(\ua+\wa,\ub+\wb,\uc+\wc)};
    \node[below] at (axis cs:\va,\vb,\vc) {$\vec v$};
    \addplot3[quiver={u=\wa,v=\wb,w=\wc},blue,-stealth,thick] 
    coordinates {(0,0,0)(\va,\vb,\vc)(\ua,\ub,\uc)(\va+\ua,\vb+\ub,\vc+\uc)};
    \node[below] at (axis cs:\wa,\wb,\wc) {$\vec w$};
  \ifnum0<#1
    \addplot3[quiver={u=0,v=0,w=\vw},red,-stealth,thick] 
    coordinates {(0,0,0)};
    \node[right] at (axis cs:0,0,\vw) {$\vec v\times\vec w$};
    \node[right] at (axis cs:0,0,0.5) {$\!\theta$};
    \addplot3[gray] coordinates {(0,0,\uc)(\ua,\ub,\uc)};
  \fi
\end{axis}
\end{tikzpicture}
}}


\begin{wrapfigure}r{0pt} \pppd0\end{wrapfigure}
Consider the \idx{parallelepiped} with edges formed by three vectors~\uv, \vv, and~\wv\ in~\(\RR^3\), as illustrated in stereo to the right.
Our challenge is to derive that the volume of the parallelepiped is~\(|\uv\cdot(\vv\times\wv)|\).

\index{volume, parallelepiped}%
Let's use that we know the volume of the parallelepiped is the area of its base times its height.
\begin{itemize}
\item The base of the parallelepiped is the \idx{parallelogram} formed with edges~\vv\ and~\wv.
Hence the base has area~\(|\vv\times\wv|\) (\cref{thm:cpgd}).

\item 
\begin{figbox}{\pppd1}%
The height of the parallelepiped is then that part of~\uv\ in the direction of a \idx{normal vector} to~\vv\ and~\wv.
We know that \(\vv\times\wv\) is orthogonal to both~\vv\ and~\wv\ (\cref{thm:cpga}), so by trigonometry the height must be \(|\uv|\cos\theta\) for angle~\(\theta\) between~\uv\ and \(\vv\times\wv\), as illustrated.
\end{figbox}

To cater for cases where \(\vv\times\wv\) points in the opposite direction to that shown, the height is~\(|\uv||\cos\theta|\).
The \idx{dot product}  determines this \idx{cosine} (\cref{thm:anglev}):
\begin{equation*}
\cos\theta=\frac{\uv\cdot(\vv\times\wv)}{|\uv||\vv\times\wv|}\,.
\end{equation*}
The height of the parallelepiped is then
\begin{equation*}
|\uv||\cos\theta|=|\uv|\frac{|\uv\cdot(\vv\times\wv)|}{|\uv||\vv\times\wv|}
=\frac{|\uv\cdot(\vv\times\wv)|}{|\vv\times\wv|}\,.
\end{equation*}
\end{itemize}
Consequently, the volume of the parallelepiped equals
\begin{equation*}
\text{base}\cdot\text{height}
=|\vv\times\wv|\frac{|\uv\cdot(\vv\times\wv)|}{|\vv\times\wv|}
=|\uv\cdot(\vv\times\wv)|.
\end{equation*}



\begin{definition} \label{def:sctrpr}
For every three vectors \uv, \vv, and~\wv\ in~\(\RR^3\), the \bfidx{scalar triple product}\index{triple product, scalar} is \(\uv\cdot(\vv\times\wv)\).
\end{definition}


\needlines8
\begin{wrapfigure}r{0pt}
\qview{30}{34}{\begin{tikzpicture} 
\def\ua{0}\def\ub{2}\def\uc{1}
\def\va{-2}\def\vb{0}\def\vc{1}
\def\wa{2}\def\wb{2}\def\wc{1}
\begin{axis}[footnotesize,font=\footnotesize,axis equal,view={\q}{25}
    ,xlabel={$x_1$},ylabel={$x_2$},zlabel={$x_3$},label shift={-1.5ex} 
    ]
    \addplot3[quiver={u=\ua,v=\ub,w=\uc},blue,-stealth,thick] 
    coordinates {(0,0,0)(\va,\vb,\vc)(\wa,\wb,\wc)(\va+\wa,\vb+\wb,\vc+\wc)};
    \node[below] at (axis cs:\ua,\ub,\uc) {$\vec u$};
    \addplot3[quiver={u=\va,v=\vb,w=\vc},blue,-stealth,thick] 
    coordinates {(0,0,0)(\ua,\ub,\uc)(\wa,\wb,\wc)(\ua+\wa,\ub+\wb,\uc+\wc)};
    \node[left] at (axis cs:\va,\vb,\vc) {$\vec v$};
    \addplot3[quiver={u=\wa,v=\wb,w=\wc},blue,-stealth,thick] 
    coordinates {(0,0,0)(\va,\vb,\vc)(\ua,\ub,\uc)(\va+\ua,\vb+\ub,\uc+\vc)};
    \node[below] at (axis cs:\wa,\wb,\wc) {$\vec w$};
\end{axis}
\end{tikzpicture}}
\end{wrapfigure}
\begin{example} \label{eg:stppv}
Use the \idx{scalar triple product} to find the volume of the parallelepiped formed by vectors \(\uv=(0,2,1)\), \(\vv=(-2,0,1)\) and \(\wv=(2,2,1)\)---as illustrated in stereo to the right.

\begin{solution} 
\cref{eg:apvw} found the cross product \(\vv\times\wv=-2\iv+4\jv-4\kv\)\,.
So the scalar triple product \(\uv\cdot(\vv\times\wv)=(2\jv+\kv)\cdot(-2\iv+4\jv-4\kv)=8-4=4\)\,.
Hence the volume of the parallelepiped is~\(4\) (cubic units). 

The order of the vectors in a scalar triple product only affects the sign of the result.  
For example, we also find the volume of this parallelepiped via~\(\vv\cdot(\uv\times\wv)\).
Returning to the procedure of \cref{eg:nviax} to find the cross product gives
\setlength{\unitlength}{1.2ex}
\def\abc#1{\begin{vmatrix}\begin{picture}(5.3,6)
%\put(0,0){\framebox(5,5){}}
\put(0,4){$\iv$}\put(2,4){$0$}\put(4,4){$2$}
\put(0,2){$\jv$}\put(2,2){$2$}\put(4,2){$2$}
\put(0,0){$\kv$}\put(2,0){$1$}\put(4,0){$1$}
\ifnum1=#1\put(0.5,-0.5){\line(0,1)6}\put(-0.5,4.5){\line(1,0)6}\fi
\ifnum2=#1\put(0.5,-0.5){\line(0,1)6}\put(-0.5,2.5){\line(1,0)6}\fi
\ifnum3=#1\put(0.5,-0.5){\line(0,1)6}\put(-0.5,0.5){\line(1,0)6}\fi
\end{picture}\end{vmatrix}}
\def\ab#1#2#3#4{\begin{vmatrix}\begin{picture}(3,4)
\put(0,2){$#1$}\put(2,2){$#2$}
\put(0,0){$#3$}\put(2,0){$#4$}
\color{red}\put(-0.5,-0.5){\line(1,1)4}
\color{blue}\put(-0.5,3.5){\line(1,-1)4}
\end{picture}\end{vmatrix}}
\begin{eqnarray*}
\uv\times\wv&=& \abc0 
%\\&&\parbox{20em}{(cross out 1st column and each row, multiplying each by common entry, with alternating sign)}
\\&=&\iv\abc1-\jv\abc2+\kv\abc3
\\&=&\iv\begin{vmatrix} 2&2\\1&1 \end{vmatrix}
-\jv\begin{vmatrix} 0&2\\1&1 \end{vmatrix}
+\kv\begin{vmatrix} 0&2\\2&2 \end{vmatrix}
%\\&&\parbox{20em}{(draw diagonals, then subtract product of red diagonal from product of the blue)}
\\&=&\iv\ab2211
-\jv\ab0211
+\kv\ab0222
\\&=&\iv(2\cdot1-1\cdot2)
-\jv(0\cdot1-1\cdot2)
+\kv(0\cdot2-2\cdot2)
\\&=&2\jv-4\kv\,.
\end{eqnarray*}
Then the triple product \(\vv\cdot(\uv\times\wv)=(-2\iv+\kv)\cdot(2\jv-4\kv)=0+0-4=-4\)\,.
Hence the volume of the parallelepiped is~\(|-4|=4\) as before.
\end{solution}
\end{example}



Using the procedure of \cref{eg:nviax} to find a \idx{scalar triple product} establishes a strong connection to the matrix \idx{determinant}s of \cref{ch:ddm}.
\index{i@$\iv$}\index{j@$\jv$}\index{k@$\kv$}%
In the second solution to the previous \cref{eg:stppv}, in finding \(\uv\times\wv\), the unit vectors~\iv, \jv, and~\kv\  just acted as place-holding symbols to eventually ensure a multiplication by the correct component of~\vv\ in the dot product.
We could seamlessly combine the two products by replacing the symbols~\iv, \jv, and~\kv\ directly with the corresponding component of~\vv:
{%%%%%%
\setlength{\unitlength}{1.6ex}
\def\abc#1{\begin{vmatrix}\ \begin{picture}(5.3,6)
%\put(0,0){\framebox(5,5){}}
\put(0,4){$\!\!\!-2$}\put(2,4){$0$}\put(4,4){$2$}
\put(0,2){$0$}\put(2,2){$2$}\put(4,2){$2$}
\put(0,0){$1$}\put(2,0){$1$}\put(4,0){$1$}
\ifnum1=#1\put(0.5,-0.5){\line(0,1)6}\put(-0.5,4.5){\line(1,0)6}\fi
\ifnum2=#1\put(0.5,-0.5){\line(0,1)6}\put(-0.5,2.5){\line(1,0)6}\fi
\ifnum3=#1\put(0.5,-0.5){\line(0,1)6}\put(-0.5,0.5){\line(1,0)6}\fi
\end{picture}\end{vmatrix}}
\def\ab#1#2#3#4{\begin{vmatrix}\begin{picture}(3,4)
\put(0,2){$#1$}\put(2,2){$#2$}
\put(0,0){$#3$}\put(2,0){$#4$}
\color{red}\put(-0.5,-0.5){\line(1,1)4}
\color{blue}\put(-0.5,3.5){\line(1,-1)4}
\end{picture}\end{vmatrix}}
\begin{eqnarray*}
\vv\cdot(\uv\times\wv)&=& \abc0 
%\\&&\parbox{20em}{(cross out 1st column and each row, multiplying each by common entry, with alternating sign)}
\\&=&-2\abc1-0\abc2+1\abc3
\\&=&-2\begin{vmatrix} 2&2\\1&1 \end{vmatrix}
-0\begin{vmatrix} 0&2\\1&1 \end{vmatrix}
+1\begin{vmatrix} 0&2\\2&2 \end{vmatrix}
%\\&&\parbox{20em}{(draw diagonals, then subtract product of red diagonal from product of the blue)}
\\&=&-2\ab2211
-0\ab0211
+1\ab0222
\\&=&-2(2\cdot1-1\cdot2)
-0(0\cdot1-1\cdot2)
+1(0\cdot2-2\cdot2)
\\&=&-2\cdot0-0(-2)+1(-4)=-4\,.
\end{eqnarray*}
}%%%%%%%%%%%%%%%%%%%%
Hence the parallelepiped formed by~\uv, \vv, and~\wv\ has volume~\(|-4|\), as before.
Here the volume follows from the above manipulations of the matrix of numbers formed with columns of the matrix being the vectors~\uv, \vv, and~\wv.
\cref{ch:ddm} shows that this computation of volume generalizes to determining, via analogous matrices of vectors, the `volume' of objects formed by vectors with any number \text{of components.}


\index{parallelepiped volume|)}


\index{cross product|)}




\sectionExercises


\begin{exercise} \label{ex:cpijk} 
Use \cref{def:cp} to establish some of the \idx{standard unit vector} identities in \cref{eg:cpijk}: 
\index{i@$\iv$}\index{j@$\jv$}\index{k@$\kv$}%
\begin{enumerate}
\item \(\jv\times\kv=\iv\)\,,\quad \(\kv\times\jv=-\iv\)\,,\quad \(\jv\times\jv=\ov\)\,;
\item \(\kv\times\iv=\jv\)\,,\quad \(\iv\times\kv=-\jv\)\,,\quad \(\kv\times\kv=\ov\)\,.
\end{enumerate}
\end{exercise}





\begin{exercise}  
Use \cref{def:cp}, perhaps via the procedure used in \cref{eg:nviax}, to determine the following cross products. 
Confirm that each cross product is orthogonal to the two vectors in the given product.
Show your details.
% u=0+round(randn(1,3)*3), v=0+round(randn(1,3)*3), uv=cross(u,v)
\begin{Parts}
\item \((3\iv+\jv)\times(3\iv -3\jv -2\kv)\)
\answer{\(-2\iv +6\jv -12\kv\)}
\item \((3\iv  +\kv)\times(5\iv +6 \kv)\)
\answer{\(-13\jv\)}
\begin{OmitV1}
\item \((2\iv-\jv -3\kv)\times(3\iv +2\kv)\)
\answer{\(-2\iv -13\jv +3\kv\)}
\item \((\iv -\jv +2\kv)\times(3\iv  +3\kv)\)
\answer{\(-3\iv +3\jv +3\kv\)}
\item \((-1,3,2)\times(3,-5,1)\)
\answer{\((13,7,-4)\)}
\item \((3,0,4)\times(5,1,2)\)
\answer{\((-4,14,3)\)}
\end{OmitV1}
\item \((4,1,3)\times(3,2,-1)\)
\answer{\((-7,13,5)\)}
\item \((3,-7,3)\times(2,1,0)\)
\answer{\((-3,6,17)\)}
\end{Parts}
\end{exercise}





\begin{exercise}  
For each of the stereo pictures below, estimate the \idx{area} of the pictured \index{parallelogram area}parallelogram by estimating the edge vectors~\vv\ and~\wv\ (all components are integers), then computing their cross product.
% u=0+round(randn(1,3)*2), v=0+round(randn(1,3)*2), uv=cross(u,v), norm(uv)
\newcommand{\temp}[6]{\qview{30}{34}{
\begin{tikzpicture} 
\begin{axis}[footnotesize,font=\footnotesize,axis equal,view={\q}{25}
    ,xlabel={$x_1$},ylabel={$x_2$},zlabel={$x_3$},label shift={-1.5ex} ]
    \threev[above]{#1}{#2}{#3}{\vec v};
    \addplot3[quiver={u=#1,v=#2,w=#3},blue,-stealth,thick] 
    coordinates {(#4,#5,#6)};
    \threev[above]{#4}{#5}{#6}{\vec w};
    \addplot3[quiver={u=#4,v=#5,w=#6},blue,-stealth,thick] 
    coordinates {(#1,#2,#3)};
    \addplot3[gray] coordinates 
    {((#1)+(#4),(#2)+(#5),0)((#1)+(#4),(#2)+(#5),#3+#6)};
\end{axis}
\end{tikzpicture}}}
\begin{enumerate}
\item \temp{2}{-4}{0}{2}{0}{-2}
\answer{\(12\)}

\item \temp{3}{2}{1}{3}{2}{0}
\answer{\(\sqrt{13}=3.606\)}

\begin{OmitV1}
\item \temp{0}{4}{2}{3}{0}{1}
\answer{\(14\)}

\item \temp{3}{4}{1}{3}{-1}{0}
\answer{\(\sqrt{235}=15.33\)}
\end{OmitV1}

\item \temp{-3}{1}{-2}{-1}{3}{1}
\answer{\(\sqrt{138}=11.75\)}

\item \temp{4}{-3}{1}{1}{0}{2}
\answer{\(\sqrt{94}=9.695\)}

\end{enumerate}
\end{exercise}





\begin{exercise}  
Each of the following equations describes a plane in 3D.
Find a \idx{normal vector} to each of the planes.
%pqr=0+round(randn(3)*2),n=cross(pqr(2,:),pqr(3,:))
\begin{enumerate}
\item \(\xv=(-1,0,1)+(-5,2,-1)s+(2,-4,0)t\)
\answer{\(\propto(-2,-1,8)\)}

\item \(2x+2y+4z=20\)
\answer{\(\propto(1,1,2)\)}

\begin{OmitV1}
\item \(x_1-x_2+x_3+2=0\)
\answer{\(\propto\iv-\jv+\kv\)}

\item \(\xv=6\iv-3\jv+(3\iv-3\jv-2\kv)s-(\iv+\jv+\kv)\)
\answer{\(\propto\iv+5\jv-6\kv\)}
\end{OmitV1}

\item \(\xv=\jv+2\kv+(\iv-\kv)s+(-5\iv+\jv-3\kv)t\)
\answer{\(\propto\iv+8\jv+\kv\)}

\item \(3y=x+2z+4\)
\answer{\(\propto -\iv+3\jv-2\kv\)}

\begin{OmitV1}
\item \(3p+8q-9=4r\)
\answer{\((p,q,r)\propto(3,8,-4)\)}

\item \(\xv=(-2,2,-3)+(-3,2,0)s+(-1,3,2)t\)
\answer{\(\propto(4,6,-7)\)}
\end{OmitV1}

\end{enumerate}
\end{exercise}







\begin{exercise} \label{ex:cpga} 
Use \cref{def:cp} to prove that, for all vectors \(\vv,\wv\) in~\(\RR^3\), the cross product~\(\vv\times\wv\)~is orthogonal to~\wv.
\end{exercise}




\begin{exercise} \label{ex:cpgc} 
Prove the identity that for every pair of vectors \(\vv,\wv\) in~\(\RR^3\), \(|\vv\times\wv|^2=|\vv|^2|\wv|^2-(\vv\cdot\wv)^2\) (an identity invoked in the proof of \cref{thm:cpgc}). 
Use the algebraic \cref{def:dotprod,def:cp} of the dot and cross products to expand both sides of the identity and show that both sides expand to the same complicated expression.
\end{exercise}




\begin{exercise}  
Using \cref{thm:cpp}, and the identities among \idx{standard unit vector}s of \cref{eg:cpijk}, compute the following cross products.
Record and justify each step in detail.
% uv=0+round(randn(2,3).^2),cross(uv(1,:),uv(2,:))
\begin{Parts}
\item \(\iv\times(3\jv)\)
\answer{\(3\kv\)}
\item \((4\jv+3\kv)\times\kv\)
\answer{\(4\iv\)}
\begin{OmitV1}
\item \((4\kv)\times(\iv+6\jv)\)
\answer{\(-24\iv +4\jv\)}
\item \(\jv\times(3\iv+2\kv)\)
\answer{\(2\iv-3\kv\)}
\end{OmitV1}
\item \((2\iv+2\kv)\times(\iv+\jv)\)
\answer{\(-2\iv +2\jv +2\kv\)}
\item \((\iv-5\jv)\times(-\jv+3\kv)\)
\answer{\(-15\iv-3\jv-\kv\)}
%\item \(()\times()\)
%\answer{\(\iv \jv \kv\)
%\end{answer}
\end{Parts}
\end{exercise}




\begin{exercise}  
You are given that three specific vectors~\uv, \vv, and~\wv\ in~\(\RR^3\) have the following cross products:
% uvw=0+round(randn(3,3)*2); uv=cross(uvw(1,:),uvw(2,:)), uw=cross(uvw(1,:),uvw(3,:)), vw=cross(uvw(2,:),uvw(3,:))
\begin{align*}
&\uv\times\vv=-\jv+\kv\,,
&&\uv\times\wv=\iv-\kv\,,
&&\vv\times\wv=-\iv+2\jv\,.
\end{align*}
Use \cref{thm:cpp} to compute the following cross products.
Record and justify each step in detail.
%cs=0+round(randn(2,3).^2), xy=cs*uvw; cross(xy(1,:),xy(2,:))

\begin{Parts}
\item \((\uv+\vv)\times\wv\)
\answer{\(2\jv -\kv\)}
\item \((3\uv+\wv)\times(2\uv)\)
\answer{\(-2\iv+2\kv\)}
\begin{OmitV1}
\item \((3\vv)\times(\uv+\vv)\)
\answer{\(3\jv -3\kv\)}
\item \((2\vv+\wv)\times(\uv+3\vv)\)
\answer{\(2\iv -4\jv -\kv\)}
\end{OmitV1}
\item \((2\vv+3\wv)\times(\uv+2\wv)\)
\answer{\(-7\iv +10\jv +\kv\)}
\item \((\uv+4\vv+2\wv)\times\wv\)
\answer{\(-3\iv +8\jv -\kv\)}
%\item \(()\times()\)
%\answer{\(\iv \jv \kv\)}
\end{Parts}
\end{exercise}







\begin{exercise} \label{ex:cppa} 
Use \cref{def:cp} to algebraically prove \cref{thm:cppa}---the property that \(\wv\times\vv=-(\vv\times\wv)\).
Explain how this property also follows from the basic geometry of the cross product (\cref{thm:cpg}).
\end{exercise}



\begin{exercise} \label{ex:cppb} 
Use \cref{def:cp} to algebraically prove \cref{thm:cppb}---the property that \((c\vv)\times\wv=c(\vv\times\wv)=\vv\times(c\wv)\).
Explain how this property also follows from the basic geometry of the cross product (\cref{thm:cpg})---consider \(c>0\), \(c=0\), and \(c<0\) separately.
\end{exercise}



\begin{exercise} \label{ex:cppc} 
Use \cref{def:cp} to algebraically prove \cref{thm:cppc}---the distributive property that \(\uv\times(\vv+\wv)=\uv\times\vv+\uv\times\wv\).
\end{exercise}





\begin{exercise}  
For each of the following illustrated \idx{parallelepiped}s:
estimate the edge vectors~\uv, \vv, and~\wv\ (all components are integers);
then use the \idx{scalar triple product} to estimate the volume of the parallelepiped\index{volume, parallelepiped}.
% uvw=0+round(randn(3)+1),printf('{%i}',uvw),vol=det(uvw)
\newcommand{\temp}[9]{\qview{30}{34}{
\begin{tikzpicture} 
\begin{axis}[footnotesize,font=\footnotesize,axis equal,view={\q}{25}
    ,xlabel={$x_1$},ylabel={$x_2$},zlabel={$x_3$},label shift={-1.5ex} 
    ]
    \addplot3[quiver={u=#1,v=#2,w=#3},blue] 
    coordinates {(#4,#5,#6)(#7,#8,#9)(#4+#7,#5+#8,#6+#9)};
    \threev{#1}{#2}{#3}{\vec u};
    \addplot3[quiver={u=#4,v=#5,w=#6},blue] 
    coordinates {(#1,#2,#3)(#7,#8,#9)(#1+#7,#2+#8,#3+#9)};
    \threev{#4}{#5}{#6}{\vec v};
    \addplot3[quiver={u=#7,v=#8,w=#9},blue] 
    coordinates {(0,0,0)(#4,#5,#6)(#1,#2,#3)(#4+#1,#5+#2,#3+#6)};
    \threev{#7}{#8}{#9}{\vec w};
\end{axis}
\end{tikzpicture}}}
\begin{enumerate}
\item  \temp{0}{2}{2}{3}{2}{1}{0}{-1}{1}
\answer{\(12\)}

\item  \temp{2}{2}{1}{-1}{0}{1}{2}{1}{0}
\answer{\(1\)}

\begin{OmitV1}
\item  \temp{2}{-1}{0}{2}{2}{1}{0}{2}{-1}
\answer{\(10\)}

\item  \temp{2}{2}{0}{1}{0}{2}{-1}{2}{0}
\answer{\(12\)}
\end{OmitV1}

\item \temp{-3}{0}{0}{0}{1}{2}{5}{-1}{0}
\answer{\(6\)}

\item  \temp{1}{3}{0}{0}{1}{0}{1}{0}{2}
\answer{\(2\)}

\end{enumerate}
\end{exercise}









\begin{exercise}  
In a few sentences, answer\slash discuss each of the following.
\begin{enumerate}
\item What properties of the cross product differ from that of the multiplication of scalar numbers?  

\item How is the cross product useful in changing from a parametric equation of a plane to a normal equation of the plane?

\item Given the properties \(\uv\cdot\vv=|\uv||\vv|\cos\theta\) and \(|\uv\times\vv|=|\uv||\vv|\sin\theta\)\,, why is the dot product more useful for determining the angle~\(\theta\) between the vectors~\uv\ and~\vv?

\end{enumerate}
\end{exercise}

\begin{comment}%{ED498555.pdf}
why, what caused X?
how did X occur?
what-if? what-if-not?
how does X compare with Y?
what is the evidence for X?
why is X important?
\end{comment}
