%!TEX root = ../larxxia.tex

\section{Summary of vectors}
\label{sec:sumv}


\begin{itemize}
\def\index#1{}% turn off indexing

\subsubsection{Vectors have magnitude and direction}

\item Quantities that have the properties of both a magnitude and a direction are called {vectors}.  
Using a coordinate system, with \(n\)~coordinate axes in~\(\RR^n\), a {vector} is an ordered \(n\)-tuple of real numbers represented as a row in parentheses or as a column in brackets (\autoref{def:vecs}):
\begin{equation*}
(\hlist vn)=\begin{bmatrix} v_1\\v_2\\\vdots\\v_n \end{bmatrix}
\end{equation*}

\item  The {length} (or {magnitude}) of vector~\vv\  is  (\autoref{def:veclen})
\begin{equation*}
|\vv|:=\sqrt{v_1^2+v_2^2+\cdots+v_n^2}\,.
\end{equation*}
A vector of length one is called a {unit vector}.
The {zero vector}~\ov\ is the only vector of {length} zero (\autoref{thm:veclen0}).




\subsubsection{Adding and stretching vectors}

\item The {sum} or {addition} of~\uv\ and~\vv, denoted \(\uv+\vv\), is the vector obtained by joining~\vv\ to~\uv\ `head-to-tail', and is computed as (\autoref{def:vecops})
\begin{equation*}
\uv+\vv:=(u_1+v_1,u_2+v_2,\ldots,u_n+v_n).
\end{equation*}
The {scalar multiplication} of~\uv\ by~\(c\) is computed as
\begin{equation*}
c\uv:=(cu_1,cu_2,\ldots,cu_n),
\end{equation*}
and has length~\(|c||\uv|\) in the direction of~\uv\ when \(c>0\) but in the opposite direction when \(c<0\).

\item The {standard unit vector}s in~\(\RR^n\), \(\hlist\ev n\), are the unit vectors in the direction of the corresponding coordinate axis (\autoref{def:stuniv}).
In \(\RR^2\) and~\(\RR^3\) they are often denoted by \(\iv=\ev_1\), \(\jv=\ev_2\) and \(\kv=\ev_3\)\,.

\item The {distance} between vectors~\uv\ and~\vv\ is the \idx{length} of their difference, \(|\uv-\vv|\) (\autoref{def:vecdist}).

\item A {parametric equation} of a line is \(\xv=\pv+t\dv\) where \pv~is any point on the line, \dv~is the {direction vector}, and the {scalar} {parameter}~\(t\) varies over all real values (\autoref{def:parlin}).

\item Addition and scalar multiplication of vectors satisfy the following familiar algebraic properties (\autoref{thm:vecops}):
\begin{itemize}
\item \(\uv+\vv=\vv+\uv\) \quad(\idx{commutative law});
\item \((\uv+\vv)+\wv=\uv+(\vv+\wv)\) \quad(\idx{associative law});
\item \(\uv+\ov=\ov+\uv=\uv\);
\item \(\uv+(-\uv)=(-\uv)+\uv=\ov\);
\item \(a(\uv+\vv)=a\uv+a\vv\)\quad(a \idx{distributive law});
\item \((a+b)\uv=a\uv+b\uv\)\quad(a \idx{distributive law});
\item \((ab)\uv=a(b\uv)\);
\item \(1\uv=\uv\);
\item \(0\uv=\ov\);
\item \(|a\uv|=|a|\cdot|\uv|\).
\end{itemize}




\subsubsection{The dot product determines angles and lengths}

\item The {dot product} (or {inner product}) of two vectors~\uv\ and~\vv\ in~\(\RR^n\) is the {scalar} (\autoref{def:dotprod})
\begin{equation*}
\uv\cdot \vv:= \lincomb uvn\,.
\end{equation*}
\begin{itemize}
\item  The {angle}~\(\theta\) between the vectors is determined by (\autoref{thm:anglev})
\begin{equation*}
\cos\theta=\frac{\uv\cdot\vv}{|\uv||\vv|}\,,
\quad 0\leq\theta\leq\pi
\quad (0\leq\theta\leq180^\circ).
\end{equation*}
\item The vectors are termed {orthogonal} (or {perpendicular}) if their dot product \(\uv\cdot\vv=0\) (\autoref{def:orthovec}).
\end{itemize}

\item The dot product (inner product) of vectors satisfy the following algebraic properties (Theorems~\ref{thm:dotops} and~\ref{thm:triscal}):
\begin{itemize}
\item \(\uv\cdot\vv=\vv\cdot\uv\) \quad({commutative law});
\item \(\uv\cdot\ov=\ov\cdot\uv=0\);
\item \(a(\uv\cdot\vv)=(a\uv)\cdot\vv=\uv\cdot(a\vv)\);
\item \((\uv+\vv)\cdot\wv=\uv\cdot\wv+\vv\cdot\wv\)\quad({distributive law});
\item \(\uv\cdot\uv\geq0\)\,, and moreover, \(\uv\cdot\uv=0\) if and only if \(\uv=\ov\)\,.
\item \(\sqrt{\uv\cdot\uv}=|\uv|\), the {length} of~\uv;
\item \(|\uv\cdot\vv|\leq|\uv||\vv|\) ({Cauchy--Schwarz inequality});
\item \(|\uv\pm\vv|\leq|\uv|+|\vv|\) ({triangle inequality}).
\end{itemize}

\item There is no non-zero vector \idx{orthogonal} to all \(n\)~{standard unit vector}s in~\(\RR^n\) (\autoref{thm:nononz}).
There can be no more than \(n\)~orthogonal unit vectors in a set of vectors in~\(\RR^n\) (\autoref{thm:orthcomp}).

\item A {parametric equation} of a plane is \(\xv=\pv+s\uv+t\vv\) for some point~\pv\ in the plane, and two vectors~\uv\ and~\vv\  parallel to the plane, and where the {scalar} {parameter}s~\(s\) and~\(t\) vary over all real values (\autoref{def:parpla}).




\subsubsection{The cross product}

\item The {cross product}  (or {vector product}) of vectors~\vv\ and~\wv\ in~\(\RR^3\) is (\autoref{def:cp})
\begin{equation*}
\vv\times\wv:=\iv(v_2w_3-v_3w_2)
+\jv(v_3w_1-v_1w_3)
+\kv(v_1w_2-v_2w_1).
\end{equation*}
\autoref{thm:cpg} gives the geometry:
\begin{itemize}
\item the vector~\(\vv\times\wv\) is {orthogonal} to both~\vv\ and~\wv;

\item the direction of~\(\vv\times\wv\) is in the right-hand sense; 

\item \(|\vv\times\wv|=|\vv|\,|\wv|\sin\theta\) where \(\theta\)~is the {angle} between vectors~\vv\ and~\wv\ (\(0\leq\theta\leq\pi\), equivalently \(0^\circ\leq\theta\leq180^\circ\)); and

\item the {length}~\(|\vv\times\wv|\) is the {area} of the parallelogram with edges~\vv\ and~\wv.
\end{itemize}

\item The cross product has the following algebraic properties (\autoref{thm:cpp}):
\begin{enumerate}
\item \(\vv\times\vv=\ov\);
\item \(\wv\times\vv=-(\vv\times\wv)\) \quad(not commutative);
\item \((c\vv)\times\wv=c(\vv\times\wv)=\vv\times(c\wv)\);
\item \(\uv\times(\vv+\wv)=\uv\times\vv+\uv\times\wv\) \quad({distributive law}).
\end{enumerate}

\item The {scalar triple product} \(\uv\cdot(\vv\times\wv)\) (\autoref{def:sctrpr}) is the volume of the parallelepiped with edges~\uv, \vv\ and~\wv.





\subsubsection{Use \script\ for vector computation}

\item \verb|[ ... ]| forms vectors: use \(n\)~numbers separated by semi-colons for vectors in~\(\RR^n\) (or use newlines instead of the semi-colons).  

\item \verb|=| assigns the result of the expression to the right of the~\verb|=| to the variable name on the left.

\item \verb|norm(v)| computes the {length}\slash {magnitude} of the vector~\verb|v| (\autoref{def:veclen}).

\item \verb|+,-,*| is vector\slash scalar addition, subtraction, and multiplication.
Parentheses \verb|()| control the order of operations.

\item \verb|/x| divides a vector\slash scalar by a scalar~\(x\).

\item \verb|x^y| for scalars~\(x\) and~\(y\) computes~\(x^y\).

\item \verb|dot(u,v)| computes the {dot product} of vectors~\verb|u| and~\verb|v| (\autoref{def:dotprod}).

\item \verb|acos(q)| computes the {arc-cos}, the {inverse cosine}, of the scalar~\verb|q| in radians.  
To find the angle in degrees use \verb|acos(q)*180/pi|\,.

\item \verb|quit| terminates the \script\ session.





\end{itemize}


\makeanswers
