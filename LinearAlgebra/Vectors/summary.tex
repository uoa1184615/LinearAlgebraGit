%!TEX root = ../larxxia.tex

\section{Summary of vectors}
\label{sec:sumv}


\begin{itemize}
\def\index#1{}% turn off indexing

\subsubsection{Vectors have magnitude and direction}

\itemme Quantities that have the properties of both a magnitude and a direction are called \idx{vectors}.  
Using a coordinate system, with \(n\)~coordinate axes in~\(\RR^n\), a \bfidx{vector} is an ordered \(n\)-tuple of real numbers represented as a row in parentheses or as a column in brackets (\autoref{def:vecs}):
\begin{equation*}
(\hlist vn)=\begin{bmatrix} v_1\\v_2\\\vdots\\v_n \end{bmatrix}
\end{equation*}

In applications, the components of vectors have physical units such as metres, or~km/hr, or numbers of words---usually the components all have the same units.

\itemme The set of all vectors with \(n\)~components is denoted~\(\RR^n\) (\autoref{def:rRn}).
The vector with all components zero,  \((0,0,\ldots,0)\), is called the \bfidx{zero vector} and denoted by~\ov.

\itemhi  The \bfidx{length} (or \bfidx{magnitude}) of vector~\vv\  is  (\autoref{def:veclen})
\begin{equation*}
|\vv|:=\sqrt{v_1^2+v_2^2+\cdots+v_n^2}\,.
\end{equation*}
A vector of length one is called a \bfidx{unit vector}.
The \bfidx{zero vector}~\ov\ is the only vector of \idx{length} zero (\autoref{thm:veclen0}).




\subsubsection{Adding and stretching vectors}

\itemme The \bfidx{sum} or \bfidx{addition} of~\uv\ and~\vv, denoted \(\uv+\vv\), is the vector obtained by joining~\vv\ to~\uv\ `head-to-tail', and is computed as (\autoref{def:vecops})
\begin{equation*}
\uv+\vv:=(u_1+v_1,u_2+v_2,\ldots,u_n+v_n).
\end{equation*}
The \bfidx{scalar multiplication} of~\uv\ by~\(c\) is computed as
\begin{equation*}
c\uv:=(cu_1,cu_2,\ldots,cu_n),
\end{equation*}
and has length~\(|c||\uv|\) in the direction of~\uv\ when \(c>0\) but in the opposite direction when \(c<0\).

\itemme The \bfidx{standard unit vector}s in~\(\RR^n\), \(\hlist\ev n\), are the unit vectors in the direction of the corresponding coordinate axis (\autoref{def:stuniv}).
In \(\RR^2\) and~\(\RR^3\) they are often denoted by \(\iv=\ev_1\), \(\jv=\ev_2\) and \(\kv=\ev_3\)\,.

\itemlo The \bfidx{distance} between vectors~\uv\ and~\vv\ is the \idx{length} of their difference, \(|\uv-\vv|\) (\autoref{def:vecdist}).

\itemlo A \bfidx{parametric equation} of a line is \(\xv=\pv+t\dv\) where \pv~is any point on the line, \dv~is the \bfidx{direction vector}, and the \idx{scalar} \bfidx{parameter}~\(t\) varies over all real values (\autoref{def:parlin}).

\itemhi Addition and scalar multiplication of vectors satisfy the following familiar algebraic properties (\autoref{thm:vecops}):
\begin{itemize}
\item \(\uv+\vv=\vv+\uv\) \quad(\idx{commutative law});
\item \((\uv+\vv)+\wv=\uv+(\vv+\wv)\) \quad(\idx{associative law});
\item \(\uv+\ov=\ov+\uv=\uv\);
\item \(\uv+(-\uv)=(-\uv)+\uv=\ov\);
\item \(a(\uv+\vv)=a\uv+a\vv\)\quad(a \idx{distributive law});
\item \((a+b)\uv=a\uv+b\uv\)\quad(a \idx{distributive law});
\item \((ab)\uv=a(b\uv)\);
\item \(1\uv=\uv\);
\item \(0\uv=\ov\);
\item \(|a\uv|=|a|\cdot|\uv|\).
\end{itemize}




\subsubsection{The dot product determines angles and lengths}

\itemhi The \bfidx{dot product} (or \bfidx{inner product}) of two vectors~\uv\ and~\vv\ in~\(\RR^n\) is the \idx{scalar} (\autoref{def:dotprod})
\begin{equation*}
\uv\cdot \vv:= \lincomb uvn\,.
\end{equation*}
\begin{itemize}
\itemhi  Determine the \bfidx{angle}~\(\theta\) between the vectors by (\autoref{thm:anglev})
\begin{equation*}
\cos\theta=\frac{\uv\cdot\vv}{|\uv||\vv|}\,,
\quad 0\leq\theta\leq\pi
\quad (0\leq\theta\leq180^\circ).
\end{equation*}
In applications, the angle between two vectors tells us whether the vectors are in a similar direction, or not.
\itemhi The vectors are termed \bfidx{orthogonal} (or \bfidx{perpendicular}) if their dot product \(\uv\cdot\vv=0\) (\autoref{def:orthovec}).
\end{itemize}

\item In mechanics the work done by a force~\Fv\ on body that moves a distance~\dv\ is the dot product \(W=\Fv\cdot\dv\).

\itemme The dot product (inner product) of vectors satisfy the following algebraic properties (Theorems~\ref{thm:dotops} and~\ref{thm:triscal}):
\begin{itemize}
\item \(\uv\cdot\vv=\vv\cdot\uv\) \quad({commutative law});
\item \(\uv\cdot\ov=\ov\cdot\uv=0\);
\item \(a(\uv\cdot\vv)=(a\uv)\cdot\vv=\uv\cdot(a\vv)\);
\item \((\uv+\vv)\cdot\wv=\uv\cdot\wv+\vv\cdot\wv\)\quad({distributive law});
\item \(\uv\cdot\uv\geq0\)\,, and moreover, \(\uv\cdot\uv=0\) if and only if \(\uv=\ov\)\,.
\item \(\sqrt{\uv\cdot\uv}=|\uv|\), the \idx{length} of~\uv;
\item \(|\uv\cdot\vv|\leq|\uv||\vv|\) ({Cauchy--Schwarz inequality});
\item \(|\uv\pm\vv|\leq|\uv|+|\vv|\) ({triangle inequality}).
\end{itemize}

\itemlo There is no non-zero vector \idx{orthogonal} to all \(n\)~{standard unit vector}s in~\(\RR^n\) (\autoref{thm:nononz}).
There can be no more than \(n\)~orthogonal unit vectors in a set of vectors in~\(\RR^n\) (\autoref{thm:orthcomp}).

\itemlo A \bfidx{parametric equation} of a plane is \(\xv=\pv+s\uv+t\vv\) for some point~\pv\ in the plane, and two vectors~\uv\ and~\vv\  parallel to the plane, and where the \idx{scalar} \idx{parameter}s~\(s\) and~\(t\) vary over all real values (\autoref{def:parpla}).




\subsubsection{The cross product}

\itemhi The \bfidx{cross product}  (or \bfidx{vector product}) of vectors~\vv\ and~\wv\ in~\(\RR^3\) is (\autoref{def:cp})
\begin{equation*}
\vv\times\wv:=\iv(v_2w_3-v_3w_2)
+\jv(v_3w_1-v_1w_3)
+\kv(v_1w_2-v_2w_1).
\end{equation*}
\autoref{thm:cpg} gives the geometry:
\begin{itemize}
\item the vector~\(\vv\times\wv\) is \idx{orthogonal} to both~\vv\ and~\wv;

\item the direction of~\(\vv\times\wv\) is in the right-hand sense; 

\item \(|\vv\times\wv|=|\vv|\,|\wv|\sin\theta\) where \(\theta\)~is the \idx{angle} between vectors~\vv\ and~\wv\ (\(0\leq\theta\leq\pi\), equivalently \(0^\circ\leq\theta\leq180^\circ\)); and

\item the \idx{length}~\(|\vv\times\wv|\) is the \idx{area} of the parallelogram with edges~\vv\ and~\wv.
\end{itemize}

\itemlo The cross product has the following algebraic properties (\autoref{thm:cpp}):
\begin{enumerate}
\item \(\vv\times\vv=\ov\);
\item \(\wv\times\vv=-(\vv\times\wv)\) \quad(not commutative);
\item \((c\vv)\times\wv=c(\vv\times\wv)=\vv\times(c\wv)\);
\item \(\uv\times(\vv+\wv)=\uv\times\vv+\uv\times\wv\) \quad({distributive law}).
\end{enumerate}

\itemlo The \bfidx{scalar triple product} \(\uv\cdot(\vv\times\wv)\) (\autoref{def:sctrpr}) is the volume of the parallelepiped with edges~\uv, \vv\ and~\wv.





\subsubsection{Use \script\ for vector computation}

\itemhi \verb|[ ... ]| forms vectors: use \(n\)~numbers separated by semi-colons for vectors in~\(\RR^n\) (or use newlines instead of the semi-colons).  

\itemhi \verb|=| assigns the result of the expression to the right of the~\verb|=| to the variable name on the left.

\itemme \verb|norm(v)| computes the \idx{length}\slash \idx{magnitude} of the vector~\verb|v| (\autoref{def:veclen}).

\itemme \verb|+,-,*| is vector\slash scalar addition, subtraction, and multiplication.
Parentheses \verb|()| control the order of operations.

\item \verb|/x| divides a vector\slash scalar by a scalar~\(x\).

\item \verb|x^y| for scalars~\(x\) and~\(y\) computes~\(x^y\).

\item \verb|dot(u,v)| computes the \idx{dot product} of vectors~\verb|u| and~\verb|v| (\autoref{def:dotprod}).

\item \verb|acos(q)| computes the \idx{arc-cos}, the \idx{inverse cosine}, of the scalar~\verb|q| in radians.  
To find the angle in degrees use \verb|acos(q)*180/pi|\,.

\itemhi \verb|quit| terminates the \script\ session.





\end{itemize}


\makeanswers
