\begin{draft}
\section{Mathematical writing}
\label{sec:write}

\secttoc

\begin{comment}
\cite{Higham98} Chapters 3 and~4: dos and don'ts; etc.
\cite{Zobel04}
\end{comment}

Some exercises ask you to ``explain'', ``describe'' or ``write a few sentences''.

\paragraph{Written mathematics is not lecturing mathematics}
Lecture and class presentations are audio-visual extravagances evolving in time.
The written word is fixed and read.
These two mediums for conveying information are different.
Consequently, written mathematics is different to presentation mathematics.

You model for how to write with mathematics should be textbooks, not your classes.



\paragraphs{Mathematical symbols have meaning in English}
nouns and verbs



\paragraph{Integrate mathematics into sentences}



\paragraph{Punctuate}



\paragraph{Prefer short sentences}



\paragraph{Words versus symbols}



\paragraph{Avoid meaningless repetition}



\paragraph{Avoid gratuitous colons}



\paragraph{Reflect the specified problem}
If a problem or exercise asks you to solve for a variable, say solve  \(3x-6=0\) for~\(x\), then ensure you finish the write-up with a statement asserting ``\(x=2\,.\)'', do not just write a bald~``\(2\)''.



\paragraph{Avoid ``can be''}
The words ``can be'' mean that something occurs with probability greater than zero and less than or equal to one.
If the thing does actually occur, then say so definitely---do not weaken the statement with ``can''.
If the thing occurs with probability less than one, then use a more meaningful word such as ``almost always'', ``usually'', ``often'', ``mostly'', ``frequently'', ``sometimes'', ``rarely'',  or ``almost never''.



\paragraph{Clarify this}
and other pronouns such as ``it''.




\paragraph{Prefer active writing to passive}








\subsection{Exercises}


\end{draft}
