%!TEX root = ../larxxia.tex

\chapter{Vectors}
\label{ch:v}
\index{vector|(}

\minitoc

\begin{comment}
Introduce vectors, lines and planes, including adapting material from parts of Chapter~1 \pooliv{pp.1--55}, Chapter~2 (pre-\S2.1) of the book by \cite{Hopcroft2014},  \holti{\S2.1}, \larsvii{\S4.1} (short), et al.
Also Chapter~13 by \cite{HughesHallett2013}.
\end{comment}



\begin{aside}
This chapter is a relatively concise introduction to vectors, their properties, and a little computation with \script.  
Skim or study as needed.
\end{aside}%
Mathematics started with counting.
The \idx{natural number}s \(1,2,3,\ldots\) quantify how many objects have been counted.
Historically, there were many existential arguments over many centuries about whether negative numbers and zero are meaningful.
Nonetheless, eventually negative numbers and the zero were included to form the \bfidx{integer}s \(\ldots,-2,-1,0,1,2,\ldots\)\,.
In the mean time people needed to quantify fractions such as two and 
half a bags, or a third of a cup which led to the \idx{rational number}s such as \(\tfrac13\) or \(2\tfrac12=\tfrac52\). 
Now \bfidx{rational number}s are defined as all numbers writeable in the form~\(\tfrac pq\) for integers~\(p\) and~\(q\) (\(q\)~nonzero).
Over two thousand years ago, \idx{Pythagoras}%
\footnote{Pythagoras was born on Samos, Greece, in 569\,\textsc{bc}, studied in Egypt, imprisoned in Babylon, then settled in Crotona to found a philosophical and religious school.  Pythagoras and his school established Pythagoras' Theorem, the sum of the angles of a triangle are two right-angles, properties of polygons and polyhedra, and solved equations geometrically.}
was forced to recognize that for many triangles the length of a side could not be rational, and hence there must be more numbers in the world about us than rationals could provide.
To cope with non-rational numbers such as~\(\sqrt2=1.41421\cdots\) and \(\pi=3.14159\cdots\), mathematicians define the \bfidx{real numbers} to be all numbers which in principle can be written as a decimal expansion such as~\(\sqrt2\), \(\pi\),
\begin{equation*}
\tfrac97=1.285714285714\cdots
\quad\text{or}\quad
e=2.718281828459\cdots\,.
\end{equation*}
Such decimal expansions may terminate or repeat or may need to continue on indefinitely (as denoted by the three dots, called an \idx{ellipsis}).
The frequently invoked symbol~\index{R@\RR|textbf}\RR\ denotes the \emph{set} of all possible real numbers.

In the sixteenth century \index{Cardano, Gerolamo}Gerolamo Cardano%
\footnote{Considered one of the great mathematicians of the Renaissance, Cardano was one of the key figures in the foundation of probability and the earliest introducer of the binomial coefficients and the binomial theorem in the western world. 
%He wrote more than 200 works on medicine, mathematics, physics, chemistry, biology, astronomy, philosophy, religion, and music. 
%\ldots\ he is well-known for his achievements in algebra. 
\ldots\ 
He made the first systematic use of negative numbers, published with attribution the solutions of other mathematicians for the cubic and quartic equations, and acknowledged the existence of imaginary numbers. 
\hfill\mbox{\emph{\idx{Wikipedia} 2015 \cite[]{Westfall2012}}}}
developed a procedure to solve cubic polynomial equations.
But the procedure involved manipulating~\(\sqrt{-1}\) which seemed a crazy figment of imagination.
Nonetheless the procedure worked.
Subsequently, many practical uses were found for~\(\sqrt{-1}\), now denoted by~\index{i@$\i$}\(\i\) (or~\(j\) in some disciplines).
Consequently, many areas of modern science and engineering use \bfidx{complex number}s which are those of the form \(a+b\i\) for real numbers~\(a\) and~\(b\).
The symbol~\index{C@\CC|textbf}\CC\ denotes the set of all possible complex numbers.
This book mostly uses integers and real numbers, but eventually (\cref{ch:gee}) we need the marvellous complex numbers.

This book uses the term \bfidx{scalar} to denote a number that could be \idx{integer}, real or complex.
In this book, and before \cref{ch:gee}, a scalar is usually real valued.
The term `scalar' arises because such numbers are often used to scale the length of a `vector'.


