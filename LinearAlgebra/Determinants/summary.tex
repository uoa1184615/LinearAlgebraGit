%!TEX root = ../larxxia.tex

\section{Summary of determinants}
\label{sec:sumd}


\begin{itemize}
\def\index#1{}% turn off indexing

\itemhi The `graphical formulas'~\eqref{eq:dets23b} for \(2\times2\) and \(3\times3\) determinants are useful for both theory and many practical small problems.


\subsubsection*{Geometry underlies determinants}

\item The term \index{nD-cube@$n$D-cube|textbf}\textbf{$n$D-cube}  generalizes a square and cube to \(n\)~dimensions (\(\RR^n)\).
The term  \index{nD-volume@$n$D-volume|textbf}\textbf{$n$D-volume} generalizes the notion of area and volume to \(n\)~dimensions.
When the dimension of the space is unspecified, then we may say \bfidx{hyper-cube} and \bfidx{hyper-volume}, respectively.

\itemhi Let \(A\) be an \(n\times n\) \idx{square matrix}, and let \(C\)~be the unit \index{nD-cube@$n$D-cube}$n$D-cube in~\(\RR^n\).
Transform the \(n\)D-cube~\(C\) by \(\xv\mapsto A\xv\) to its image~\(C'\) in~\(\RR^n\). 
Define the \bfidx{determinant} of~\(A\), denoted either~\(\det A\) (or sometimes~\(|A|\)) such that (\cref{def:detarea}):  \begin{itemize}
\item the \idx{magnitude}~\(|\det A|\) is the \index{nD-volume@$n$D-volume}$n$D-volume of~\(C'\); and 
\item the sign of~\(\det A\) to be negative iff the transformation reflects the orientation of the $n$D-cube.
\end{itemize}

\itemhi (\cref{thm:basicdet})
\begin{itemize}
\item   For every \(n\times n\) \idx{diagonal matrix}~\(D\),
the \idx{determinant} of~\(D\) is the product of the \idx{diagonal entries}: \(\det D=d_{11}d_{22}\cdots d_{nn}\)\,.
\item  For every \idx{orthogonal matrix}~\(Q\),  \(\det Q=\pm1\) (only one alternative, not both), and \(\det Q=\det(\tr Q)\).
\item   For every \(n\times n\) matrix~\(A\) and every scalar~\(k\), \(\det(kA)=k^n\det A\)\,.
\end{itemize}

\item Consider any \idx{bounded} smooth \index{nD-volume@$n$D-volume}$n$D-volume~\(C\) in~\(\RR^n\) and its image~\(C'\) after multiplication by \(n\times n\) matrix~\(A\).
Then (\cref{thm:detanyC})
\begin{equation*}
\det A=\pm\frac{\text{$n$D-volume of }C'}
{\text{$n$D-volume of }C}
\end{equation*}
with the negative sign when matrix~\(A\) changes the orientation.

\itemhi For every two \index{square matrix}\(n\times n\) matrices \(A\) and~\(B\), \(\det(AB)=\det(A)\det(B)\) (\cref{thm:detprod}).
Further, for \(n\times n\) matrices \hlist A\ell, \begin{equation*}
\det(A_1A_2\cdots A_\ell)=\det(A_1)\det(A_2)\cdots\det(A_\ell).
\end{equation*}


\itemme For every \idx{square matrix}~\(A\), \(\det(\tr A)=\det A\) (\cref{thm:dettr}).

\item For every \(n\times n\) \idx{square matrix}~\(A\), the \idx{magnitude} of its determinant \(|\det A|=\sigma_1\sigma_2\cdots\sigma_n\)\,, the product of all its \idx{singular value}s (\cref{thm:detsvd}).

\itemme A \idx{square matrix}~\(A\) is \idx{invertible} iff \(\det A\neq 0\) (\cref{thm:detinv}).
If a matrix~\(A\) is invertible, then \(\det(A^{-1})=1/(\det A)\).







\subsubsection*{Laplace expansion theorem for determinants}

\itemhi For every \(n\times n\) matrix~\(A\), the determinant has the following row and column properties (\cref{thm:ppdet}).
\begin{itemize}
\item
If \(A\) has a \idx{zero row} or \index{zero column}column, then \(\det A=0\)\,.
\item
If \(A\) has two \idx{identical rows} or two \idx{identical columns}, then  \(\det A=0\)\,.
\item
Let \(B\) be obtained by \idx{interchanging two rows} or \index{interchanging two columns}two columns of~\(A\), then \(\det B=-\det A\)\,.
\item
Let \(B\) be obtained by multiplying any one row or column of~\(A\) by a scalar~\(k\), then \(\det B=k\det A\)\,.
\end{itemize}

\item For every \(n\times n\) matrix~\(A\),
define the \((i,j)\)th~\bfidx{minor} \(A_{ij}\)~to be the \((n-1)\times(n-1)\) square matrix obtained from~\(A\) by omitting the \(i\)th~row and \(j\)th~column.  
If, except for the entry~\(a_{ij}\), the \(i\)th~row (or \(j\)th~column) of~\(A\) is all zero, then (\cref{thm:rpdet:vii})
\begin{equation*}
\det A=(-1)^{i+j}a_{ij}\det A_{ij}\,.
\end{equation*}

\item A \bfidx{triangular matrix} is a \idx{square matrix} where all entries are zero either to the lower-left of the diagonal or to the upper-right (\cref{def:trim}): an \idx{upper triangular} matrix has zeros in the lower-left; and a \idx{lower triangular} matrix has zeros in \text{the upper-right.}

\itemme For every \(n\times n\) \idx{triangular matrix}~\(A\),
the \idx{determinant} of~\(A\) is the product of the \idx{diagonal entries}, \(\det A=a_{11}a_{22}\cdots a_{nn}\) (\cref{thm:rpdet:vi}).

\item Let \(A\), \(B\) and~\(C\) be \(n\times n\) matrices.
If matrices~\(A\), \(B\) and~\(C\) are identical except for their \(i\)th~column, and that the \(i\)th~column of~\(A\) is the sum of the \(i\)th~columns of \(B\) and~\(C\), then \(\det A=\det B+\det C\) (\cref{thm:rpdet:v}).
Further, the same sum property holds when ``column'' is replaced by ``row'' throughout.

\itemme For every \(n\times n\) matrix \(A=\begin{bmatrix} a_{ij} \end{bmatrix}\) (\(n\geq2\)), and in terms of the \((i,j)\)th \idx{minor}~\(A_{ij}\), the \idx{determinant} of~\(A\) can be computed via expansion in any row~\(i\) or any column~\(j\) as, respectively, (\cref{thm:letdet})
\begin{eqnarray*}
\det A
&=&(-1)^{i+1}a_{i1}\det A_{i1}
+(-1)^{i+2}a_{i2}\det A_{i2}
\nonumber\\&&{}
+\cdots+(-1)^{i+n}a_{in}\det A_{in}
\nonumber\\&=&(-1)^{j+1}a_{1j}\det A_{1j}
+(-1)^{j+2}a_{2j}\det A_{2j}
\nonumber\\&&{}
+\cdots+(-1)^{j+n}a_{nj}\det A_{nj}\,.
\end{eqnarray*}

\itemme The determinant of every \(n\times n\) matrix expands to the sum of \index{factorial}\(n!\)~terms, where each term is~\(\pm1\) times a product of \(n\)~factors such that each factor comes from different rows and columns of the matrix (\cref{thm:detnfac}).



\end{itemize}



\makeanswers
