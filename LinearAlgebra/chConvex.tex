%!TEX root = svdLinAlg.tex

\chapter{Optimization over convex sets}

\begin{comment}
Convex sets has pedagogical value: it revises previous material in a new context, it introduces modelling techniques, it sets a systematic tone for the course, it prepares students for further courses.
\end{comment}

\begin{remark} Optimization example, linear constraints, max/min objective function, feasible region, convex set, graphical solution.
\end{remark}

\begin{definition}
 A \bfidx{convex set}~$C$ is a set of points (or region) in~$\RR^n$ such that the \idx{line segment} joining any two points in~$C$
lies completely in~$C$.
\end{definition}

\begin{remark}
Heuristic bounded, unbounded, vertices.
Recall \idx{dot product}, properties, \idx{line segment}.
\end{remark}


\begin{theorem} \label{thm:}
\begin{enumerate}
\item $\av\cdot\xv\leq b$~is a \idx{convex set} in~$\RR^{n}$.
\item The \idx{intersection} of two \idx{convex set}s is convex.
\item $A\xv\le \bv$ defines a \idx{convex set} in~$\RR^n$, as does 
the combination $A\xv\le \bv$ and $\xv\ge \ov$.
\end{enumerate}
\end{theorem}
\begin{prof} Previous notes.
\end{prof}


\begin{definition}
A point~$P$ with \idx{position vector} 
$\ovect{OP} ={\vv}$ 
is a \bfidx{vertex} of the \idx{convex set}~$C$ if ${\vv}$~cannot be written in
the form
\begin{equation*}
(1-t){\uv}+t{\wv},\quad 0\leq t\leq 1,
\end{equation*}
for any points ${\uv},{\wv}\in C$ except when ${\vv}= {\uv}$ (and
$t=0$) or ${\vv}={\wv}$ (and $t=1$).
\end{definition}

\begin{definition}
A \idx{convex set}~$C$ is \bfidx{bounded} if there is some positive number~$M$ such that for any point~$P$ in~$C$, the \idx{length}
\[
\|\ovect{OP}\|<M\,.
\]
(That is, the length of~$\ovect{OP}$ cannot be arbitrarily large
for points~$P$ in~$C$.)
\end{definition}

\begin{theorem}\label{convex}
Let $C$ be a \bfidx{convex set} 
defined by a set of linear inequalities $A\xv\le \bv$ and $\xv\ge \ov$.
If $C$~is \idx{bounded}, then the linear function 
\[
f(\hlist xn)=\lincomb cxn
\]
takes its \idx{maximum} (or \idx{minimum}) value on~$C$ at a \idx{vertex} of~$C$. If $C$~is an \idx{unbounded} \idx{convex set}, and if $f$~takes a maximum (or minimum) value on~$C$, then this maximum (or minimum) occurs at a \idx{vertex} of~$C$.
\end{theorem}

\begin{prof} Previous lecture notes.
\end{prof}

\begin{remark} 
Examples: \idx{slack variable}s, non-negative \idx{basic solution}s, recall \verb|A\| solves linear equations, use \index{condest}\verb|condest(A(:,p))| and \verb|A(:,p)\| in \script, formulation, solution, interpretation, and examples.
\end{remark}


\sectionExercises


