\documentclass[11pt,a5paper]{article}
\usepackage{amsmath,parskip,pgfplots}
\pgfplotsset{compat=newest}
%\usepgfplotslibrary{patchplots}% for "patch refines"
%\usepackage{pgfplotstable} % for some table plots
% view stereo with cross eyes, or vice-versa
%\newcommand{\qview}[2]{\foreach \q in {#2,#1}}
\newcommand{\answer}[1]{} % do nothing with answers
\renewcommand{\vec}[1]{\text{\boldmath$#1$}}
\renewcommand{\Vec}[1]{%
  \expandafter\def\csname#1v\endcsname%
  {\ensuremath{\vec #1}}}


\title{Exercise \jobname}
\author{A. J. Roberts, \today}
\date{}

\begin{document}

\maketitle

For each of the following lines in 2D, write down a parametric equation of the line as a linear combination of two vectors, one of which is multiplied by the parameter.

\newcommand{\mytemp}[4]{%
    \begin{tikzpicture} 
    \begin{axis}[small,font=\footnotesize
        ,axis equal, axis lines=middle, grid, domain=-4.5:4.5 ] 
        \addplot[blue,thick] {#2+(x-(#1))*(#4)/(#3)};
    \end{axis}
    \end{tikzpicture}
    \answer{e.g.\ $(#1,#2)+t(#3,#4)$}
}

\mytemp{-1.5}{0.5}{1}{-1.5}


\end{document}
