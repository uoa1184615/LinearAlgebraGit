\documentclass[11pt,a5paper]{article}
\usepackage{amsmath,amsfonts,parskip,enumitem,pgfplots}
\pgfplotsset{compat=newest}
%\usepgfplotslibrary{patchplots}% for "patch refines"
%\usepackage{pgfplotstable} % for some table plots
% view stereo with cross eyes, or vice-versa
%\newcommand{\qview}[2]{\foreach \q in {#2,#1}}
\newcommand{\answer}[1]{} % do nothing with answers
\renewcommand{\vec}[1]{\text{\boldmath$#1$}}
\renewcommand{\Vec}[1]{%
  \expandafter\def\csname#1v\endcsname%
  {\ensuremath{\vec #1}}}

\title{Exercise \jobname}
\author{A. J. Roberts, \today}
\date{}

\begin{document}

\maketitle

For each of the following subspaces, draw its orthogonal complement on the plot.

\newcommand{\temp}{\begin{tikzpicture}
    \begin{axis}[small,font=\footnotesize
    ,axis equal image,axis lines=middle
    ,xmax=5.5,ymax=5.5,xmin=-5,ymin=-5.2]
    \pgfmathparse{90*rand}\edef\z{\pgfmathresult}
    \addplot+[no marks,samples=2,domain=-5:4.5] ({\x*cos(\z)},{\x*sin(\z)}) node[right] {$\mathbb{A}$};
    \end{axis}
    \end{tikzpicture}}

\pgfmathsetseed{1234}% maybe set seed
\temp

\end{document}
