\documentclass[11pt,a5paper]{article}
\usepackage{amsmath,parskip,enumitem,pgfplots}
\pgfplotsset{compat=newest}
%\usepgfplotslibrary{patchplots}% for "patch refines"
%\usepackage{pgfplotstable} % for some table plots
% view stereo with cross eyes, or vice-versa
%\newcommand{\qview}[2]{\foreach \q in {#2,#1}}
\newcommand{\answer}[1]{} % do nothing with answers
\renewcommand{\vec}[1]{\text{\boldmath$#1$}}
\renewcommand{\Vec}[1]{%
  \expandafter\def\csname#1v\endcsname%
  {\ensuremath{\vec #1}}}

\title{Exercise \jobname}
\author{A. J. Roberts, \today}
\date{}

\begin{document}

\maketitle

\Vec u\Vec v\def\proj{\operatorname{proj}}
For each pair of vectors, draw the orthogonal projection \(\proj_\uv(\vv)\).

\newcommand{\projuv}[9]{\begin{tikzpicture}
  \begin{axis}[small,font=\footnotesize
  ,axis equal ,axis x line=none , axis y line=none
  ,samples=2, enlarge x limits={value=0.15,auto} ]
  \addplot[black,mark=*]coordinates {(0,0)};
  \addplot[red,quiver={u=#1,v=#2},-stealth]coordinates {(0,0)};
  \node[right] at (axis cs:#1,#2) {$\vec #9$};
  \addplot[blue,quiver={u=#3,v=#4},-stealth]coordinates {(0,0)};
  \node[right] at (axis cs:#3,#4) {$\vec #8$};
  \ifnum#7>0
  \addplot[red,mark=*] coordinates {(#1,#2)(#5,#6)};
  \node[right] at (axis cs:#5,#6) {$\proj_{\vec #8}(\vec #9)$};
  \addplot[brown,thick,quiver={u=#5,v=#6},-stealth]coordinates {(0,0)};
  \fi
  \end{axis}
  \end{tikzpicture}}


\projuv{-1.2}{-0.9}{ 1.8}{-1.2}{-0.41538}{ 0.27692}0uv

\end{document}
