\documentclass[11pt,a5paper]{article}
\usepackage{amsmath,parskip,enumitem,pgfplots}
\pgfplotsset{compat=newest}
%\usepgfplotslibrary{patchplots}% for "patch refines"
%\usepackage{pgfplotstable} % for some table plots
% view stereo with cross eyes, or vice-versa
%\newcommand{\qview}[2]{\foreach \q in {#2,#1}}
\newcommand{\answer}[1]{} % do nothing with answers
\renewcommand{\vec}[1]{\text{\boldmath$#1$}}
\renewcommand{\Vec}[1]{%
  \expandafter\def\csname#1v\endcsname%
  {\ensuremath{\vec #1}}}


\title{Exercise \jobname}
\author{A. J. Roberts, \today}
\date{}

\begin{document}

\maketitle

\Vec v
Recall how Example~3.3.33 introduced that finding a singular vector and singular value of a matrix~\(A\) came from maximizing~\(|A\vv|\).
Each of the following matrices, say~\(A\) for discussion, has plotted \(A\vv\) (red) adjoined to the corresponding unit vector~\(\vv\) (blue).
For each case:
\begin{enumerate}[label=\roman*.]
\item by inspection of the plot, estimate a singular vector~\(\vv_1\) that appears to maximize~\(|A\vv_1|\) (to one decimal place say);
\item estimate the corresponding singular value~\(\sigma_1\) by measuring~\(|A\vv_1|\) on the plot;
\item set the second singular vector~\(\vv_2\) to be orthogonal to~\(\vv_1\) by swapping components, and making one negative;
\item estimate the corresponding singular value~\(\sigma_2\) by measuring~\(|A\vv_2|\) on the plot;
\item compute the matrix-vector products \(A\vv_1\) and~\(A\vv_2\), and confirm that they are orthogonal (approximately).
\end{enumerate}

\newcommand{\eRose}[4]{\begin{tikzpicture}%
    %[baseline={([yshift={-\ht\strutbox}]current bounding box.north)}] 
    \begin{axis}[small,font=\footnotesize
        ,axis equal image, axis lines=middle
        ,samples=33] %25 or 33 not bad 
        \addplot[domain=0:360,quiver={u=cos(\x),v=sin(\x)},blue,-stealth] 
        ({0},{0});
        \addplot[domain=0:360,quiver={u=#1*x+#2*y,v=#3*x+#4*y},red,-stealth] 
        ({cos(\x)},{sin(\x)});
    \end{axis}
    \end{tikzpicture}}


\(A=\begin{bmatrix} 1&1\\0.2&1.4 \end{bmatrix}\) \eRose{1}{1}{0.2}{1.4}
\answer{\(\vv_1\approx(0.4,0.9)\), \(\sigma_1\approx1.9\), 
\(\vv_2\approx(0.9,-0.4)\), \(\sigma_2\approx0.6\).}

\end{document}
