\documentclass[11pt,a5paper]{article}
\usepackage{amsmath,parskip,pgfplots}
\pgfplotsset{compat=newest}
%\usepgfplotslibrary{patchplots}% for "patch refines"
%\usepackage{pgfplotstable} % for some table plots
% view stereo with cross eyes, or vice-versa
%\newcommand{\qview}[2]{\foreach \q in {#2,#1}}
\newcommand{\answer}[1]{} % do nothing with answers
\renewcommand{\vec}[1]{\text{\boldmath$#1$}}
\renewcommand{\Vec}[1]{%
  \expandafter\def\csname#1v\endcsname%
  {\ensuremath{\vec #1}}}


\title{Exercise \jobname}
\author{A. J. Roberts, \today}
\date{}

\begin{document}

\maketitle

The following graphs illustrate the transformation of the unit square through multiplying by some different matrices.
Using Theorem~3.2.48(f), which transformations appear to be that of multiplying by an orthogonal matrix?

\newcommand{\TwoD}[4]{%
    \pgfmathparse{abs(#1)+abs(#2)+abs(#3)+abs(#4)}%
    \pgfmathparse{(\pgfmathresult>5)+(\pgfmathresult>3.6)*0.5+0.5}%
    \edef\xtdst{\pgfmathresult}%
    \begin{tikzpicture}
    \begin{axis}[small,font=\footnotesize
    ,height=100pt,xtick distance={\xtdst}
      ,axis lines=middle,thick,axis equal image, enlargelimits=0.05 %,grid,no marks
      ] 
    \addplot coordinates {(0,1)(1,1)(1,0)(0,0)(0,1)(0.5,1.2)(1,1)}; 
    \addplot coordinates {(#2,#4)(#1+#2,#3+#4)(#1,#3)(0,0)
      (#2,#4)(#1*0.5+#2*1.2,#3*0.5+#4*1.2)(#1+#2,#3+#4)}; 
    \end{axis}
    \end{tikzpicture}
}


\TwoD{-1.67}{1.54}{1.07}{2.11}
\answer{No---the square is stretched.}


\end{document}
