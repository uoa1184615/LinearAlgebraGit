\documentclass[11pt,a5paper]{article}
\usepackage{amsmath,amsfonts,parskip,enumitem,pgfplots}
\pgfplotsset{compat=newest}
%\usepgfplotslibrary{patchplots}% for "patch refines"
%\usepackage{pgfplotstable} % for some table plots
% view stereo with cross eyes, or vice-versa
%\newcommand{\qview}[2]{\foreach \q in {#2,#1}}
\newcommand{\answer}[1]{} % do nothing with answers
\renewcommand{\vec}[1]{\text{\boldmath$#1$}}
\renewcommand{\Vec}[1]{%
  \expandafter\def\csname#1v\endcsname%
  {\ensuremath{\vec #1}}}

\title{Exercise \jobname}
\author{A. J. Roberts, \today}
\date{}

\begin{document}

\maketitle

\def\RR{\mathbb R}
Consider the following illustrated transformations of~\(\RR^3\).
Which of these \emph{cannot} be that of a linear transformation? 


\def\unithouseviews{33,29}%default stereo---now x-eyed
\def\unithouselabels%
{xlabel={$x_1$},ylabel={$x_2$},zlabel={$x_3$},label shift={-2ex}}

\newcommand{\ThreeDgen}[1]{\mbox{% 
    \foreach \z in \unithouseviews {% stereo pair in 3D
    \begin{tikzpicture}
    \begin{axis}[ view={\z}{30},\unithouselabels
      ,axis equal image ,no marks %,grid
      ,height=5cm ,small
      ] 
    \addplot3+[thick] table[row sep=\\] {
       0.00   0.00   0.00 \\
       1.00   0.00   0.00 \\
       1.00   0.40   0.00 \\
       1.00   0.40   0.60 \\
       1.00   0.70   0.60 \\
       1.00   0.70   0.00 \\
       1.00   1.00   0.00 \\
       0.00   1.00   0.00 \\
       0.00   0.00   0.00 \\
       0.00   0.00   1.00 \\
       1.00   0.00   1.00 \\
       0.50   0.50   1.20 \\
       0.00   0.00   1.00 \\
       0.00   1.00   1.00 \\
       0.50   0.50   1.20 \\
       1.00   1.00   1.00 \\
       0.00   1.00   1.00 \\
       0.00   1.00   0.00 \\
       1.00   1.00   0.00 \\
       1.00   1.00   1.00 \\
       1.00   0.00   1.00 \\
       1.00   0.00   0.00 \\
      }; 
    \addplot3+[thick] table[row sep=\\] {#1}; 
    \end{axis}
    \end{tikzpicture}
    }}%end stereo pair
}% end new command



\ThreeDgen{
   0.00   0.00   0.00 \\
   0.76   0.30   0.56 \\
   0.99   1.08   0.68 \\
   0.82   1.18   1.61 \\
   0.86   1.73   1.84 \\
   1.15   1.66   0.77 \\
   1.32   2.24   0.87 \\
  -0.04   1.07   0.53 \\
   0.00   0.00   0.00 \\
   0.10   0.21   1.22 \\
   0.79   0.55   1.81 \\
   0.14   1.06   2.42 \\
   0.10   0.21   1.22 \\
  -0.68   1.09   2.50 \\
   0.14   1.06   2.42 \\
   0.61   2.29   2.86 \\
  -0.68   1.09   2.50 \\
  -0.04   1.07   0.53 \\
   1.32   2.24   0.87 \\
   0.61   2.29   2.86 \\
   0.79   0.55   1.81 \\
   0.76   0.30   0.56 \\
}
\answer{Not an LT.}




\end{document}
