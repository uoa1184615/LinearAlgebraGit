\documentclass[11pt,a5paper]{article}
\usepackage{amsmath,parskip,enumitem,pgfplots}
\pgfplotsset{compat=newest}
%\usepgfplotslibrary{patchplots}% for "patch refines"
%\usepackage{pgfplotstable} % for some table plots
% view stereo with cross eyes, or vice-versa
%\newcommand{\qview}[2]{\foreach \q in {#2,#1}}
\newcommand{\answer}[1]{} % do nothing with answers
\renewcommand{\vec}[1]{\text{\boldmath$#1$}}
\renewcommand{\Vec}[1]{%
  \expandafter\def\csname#1v\endcsname%
  {\ensuremath{\vec #1}}}


\title{Exercise \jobname}
\author{A. J. Roberts, \today}
\date{}

\begin{document}

\maketitle

For each of the following networks:
\begin{itemize}
\item label the nodes; 
\item construct the symmetric adjacency matrix~\(A\) such that \(a_{ij}\)~is one if node~\(i\) is linked to node~\(j\), and \(a_{ij}\)~is zero otherwise (and zero on the diagonal);
\item in Matlab/Octave use \verb|eig()| to find all eigenvalues and eigenvectors;
\item rank the `importance' of the nodes from the magnitude of their component in the eigenvector corresponding to the largest (most positive) eigenvalue.
\end{itemize}
 

\newcommand{\netwk}[2]{\begin{tikzpicture}
    \begin{axis}[axis equal image, axis lines=none,small]
    \addplot[blue,only marks,mark=*,samples=#1+1,domain=0:360]  ({cos(x)},{sin(x)});
    \foreach \p in {1,2,...,#2} {
        \addplot[blue,thick,samples=2,domain={rnd}:{rnd}]   
        ({cos(360*round(x*#1)/#1)},{sin(360*round(x*#1)/#1)});
        }
    \end{axis}
    \end{tikzpicture}}
    
    
\pgfmathsetseed{1234}% set seed 
\netwk5{12}





\end{document}
