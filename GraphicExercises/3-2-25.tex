\documentclass[11pt,a5paper]{article}
\usepackage{amsmath,parskip,pgfplots}
\pgfplotsset{compat=newest}
%\usepgfplotslibrary{patchplots}% for "patch refines"
%\usepackage{pgfplotstable} % for some table plots
% view stereo with cross eyes, or vice-versa
%\newcommand{\qview}[2]{\foreach \q in {#2,#1}}
\newcommand{\answer}[1]{} % do nothing with answers
\renewcommand{\vec}[1]{\text{\boldmath$#1$}}
\renewcommand{\Vec}[1]{%
  \expandafter\def\csname#1v\endcsname%
  {\ensuremath{\vec #1}}}


\title{Exercise \jobname}
\author{A. J. Roberts, \today}
\date{}

\begin{document}

\maketitle

The following stereo pairs illustrate the transformation of the unit cube through multiplying by some different matrices.
Using Theorem~4.2.28(f), which transformations appear to be that of multiplying by an orthogonal matrix?

\def\unithouseviews{33,29}%default stereo---now x-eyed
\def\unithouselabels%
{xlabel={$x_1$},ylabel={$x_2$},zlabel={$x_3$},label shift={-2ex}}

\newcommand{\ThreeD}[9]{\mbox{% 
    \foreach \z in \unithouseviews {% stereo pair in 3D
    \begin{tikzpicture}
    \begin{axis}[ view={\z}{30},\unithouselabels
      ,axis equal image ,no marks %,grid
      ,height=5cm, small
      ] 
    \addplot3+[thick] coordinates {(0,0,0)(1,0,0)
      (1,0.4,0)(1,0.4,0.6)(1,0.7,0.6)(1,0.7,0)% doorway
      (1,1,0)(0,1,0)(0,0,0)(0,0,1)(1,0,1)(0.5,0.5,1.2)
      (0,0,1)(0,1,1)(0.5,0.5,1.2)(1,1,1)(0,1,1)
      (0,1,0)(1,1,0)(1,1,1)(1,0,1)(1,0,0)
      }; 
    \addplot3+[thick] coordinates {(0,0,0) (#1,#4,#7)
      (#1+0.4*#2,#4+0.4*#5,#7+0.4*#8)
      (#1+0.4*#2+0.6*#3,#4+0.4*#5+0.6*#6,#7+0.4*#8+0.6*#9)
      (#1+0.7*#2+0.6*#3,#4+0.7*#5+0.6*#6,#7+0.7*#8+0.6*#9)
      (#1+0.7*#2,#4+0.7*#5,#7+0.7*#8)% doorway
      (#1+#2,#4+#5,#7+#8)
      (#2,#5,#8) (0,0,0) (#3,#6,#9) (#1+#3,#4+#6,#7+#9)
      (0.5*#1+0.5*#2+1.2*#3,0.5*#4+0.5*#5+1.2*#6,0.5*#7+0.5*#8+1.2*#9)
      (#3,#6,#9)  (#2+#3,#5+#6,#8+#9) 
      (0.5*#1+0.5*#2+1.2*#3,0.5*#4+0.5*#5+1.2*#6,0.5*#7+0.5*#8+1.2*#9)
      (#1+#2+#3,#4+#5+#6,#7+#8+#9) (#2+#3,#5+#6,#8+#9)
      (#2,#5,#8) (#1+#2,#4+#5,#7+#8) (#1+#2+#3,#4+#5+#6,#7+#8+#9)
      (#1+#3,#4+#6,#7+#9) (#1,#4,#7)
      }; 
    \end{axis}
    \end{tikzpicture}
}}}


\ThreeD{-0.79}{0.31}{0.54
}{-0.61}{-0.53}{-0.59
}{0.10}{-0.79}{0.60}
\answer{Yes---the cube appears rotated.}


\end{document}
