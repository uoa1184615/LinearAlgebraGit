\documentclass[11pt,a5paper]{article}
\usepackage{amsmath,amsfonts,parskip,enumitem,pgfplots}
\pgfplotsset{compat=newest}
%\usepgfplotslibrary{patchplots}% for "patch refines"
%\usepackage{pgfplotstable} % for some table plots
% view stereo with cross eyes, or vice-versa
%\newcommand{\qview}[2]{\foreach \q in {#2,#1}}
\newcommand{\answer}[1]{} % do nothing with answers
\renewcommand{\vec}[1]{\text{\boldmath$#1$}}
\renewcommand{\Vec}[1]{%
  \expandafter\def\csname#1v\endcsname%
  {\ensuremath{\vec #1}}}

\title{Exercise \jobname}
\author{A. J. Roberts, \today}
\date{}

\begin{document}

\maketitle

Which of the following illustrated transformations of the plane \emph{cannot} be that of a linear transformation?
In each illustration of a transformation~\(T\), the four corners of the blue unit square (\((0,0)\), \((1,0)\), \((1,1)\), and~\((0,1)\)), are transformed to the four corners of the red figure (\(T(0,0)\), \(T(1,0)\), \(T(1,1)\), and~\(T(0,1)\))---the `roof' of the unit square clarifies which side goes where.


\newcommand{\TwoDnotLT}[6]{%5th and 6th are the distortion
    \begin{tikzpicture}
    \pgfmathparse{abs(#1)+abs(#2)+abs(#3)+abs(#4)}%
    \pgfmathparse{(\pgfmathresult>5)+(\pgfmathresult>3.6)*0.5+0.5}%
    \edef\xtdst{\pgfmathresult}%
    \begin{axis}[grid,footnotesize,font=\footnotesize
    ,xtick distance={\xtdst}
      ,axis lines=middle,thick,axis equal image ] 
    \addplot coordinates {(0,1)(1,1)(1,0)(0,0)(0,1)(0.5,1.2)(1,1)}; 
    \addplot coordinates {(#2,#4)(#1+#2+#5,#3+#4+#6)(#1,#3)(0,0)
      (#2,#4)(#1*0.5+#5/2+#2*1.2,#3*0.5+#6/2+#4*1.2)(#1+#2+#5,#3+#4+#6)}; 
    \end{axis}
    \end{tikzpicture}}


\TwoDnotLT{-0.1}{ 3.0}{-1.4}{ 0.5}{ 0.4}{-0.4}
\answer{Not an LT.}




\end{document}
