\documentclass[11pt,a5paper]{article}
\usepackage{amsmath,parskip,pgfplots}
\pgfplotsset{compat=newest}
%\usepgfplotslibrary{patchplots}% for "patch refines"
%\usepackage{pgfplotstable} % for some table plots
% view stereo with cross eyes, or vice-versa
%\newcommand{\qview}[2]{\foreach \q in {#2,#1}}
\newcommand{\answer}[1]{} % do nothing with answers
\renewcommand{\vec}[1]{\text{\boldmath$#1$}}
\renewcommand{\Vec}[1]{%
  \expandafter\def\csname#1v\endcsname%
  {\ensuremath{\vec #1}}}


\title{Exercise \jobname}
\author{A. J. Roberts, \today}
\date{}

\begin{document}

\maketitle

Consider each of the transformations shown below that transform the from blue unit square to the red parallelogram.
They each have no coordinate axes shown because it is supposed to be some transformation in nature. 
Now impose on nature our mathematical description.
Draw approximate orthogonal coordinate axes, with origin at the common corner point, so the transformation becomes that of multiplication by the specified diagonal matrix.

\def\diag{\operatorname{diag}}
\newcommand{\TwoDx}[5]{%
    \begin{tikzpicture}
    \begin{axis}[small,font=\footnotesize
      ,axis lines=none,thick,axis equal 
      ] 
    \addplot coordinates {(0,1)(1,1)(1,0)(0,0)(0,1)(0.5,1.2)(1,1)}; 
    \addplot coordinates {(#2,#4)(#1+#2,#3+#4)(#1,#3)(0,0)
      (#2,#4)(#1*0.5+#2*1.2,#3*0.5+#4*1.2)(#1+#2,#3+#4)}; 
    #5
    \end{axis}
    \end{tikzpicture}
}%


\(\diag(1/2,2)\),
\TwoDx{1.95}{0.27}{0.27}{0.55}{}

\end{document}
